\documentstyle[twoside,11pt]{felib0}
\release{4}
\date{September 90}
\routine{DIRECT}
\title{Assembly Operation}
\begin{document}
\begin{purpose}
DIRECT constructs the steering vector to direct assembly
of a system stiffness or mass matrix.
\end{purpose}
\begin{specification}
      SUBROUTINE DIRECT(NELE,ELTOP,IELTOP,JELTOP,NF,INF,JNF,\\
C           DOFNOD,STEER,ISTEER,ITEST\\
C     INTEGER NELE,ELTOP,IELTOP,JELTOP,NF,INF,JNF,DOFNOD,\\
C           STEER,ISTEER,ITES\\
C     DIMENSION ELTOP(IELTOP,JELTOP),NF(INF,JNF),STEER(ISTEER)\\
\end{specification}
\begin{description}
DIRECT uses the element topologies and the nodal freedom
information to construct the steering vector STEER.
The array ELTOP contains the element topologies, element
type and the number nodes on the element. The element type
is stored in ELTOP(NELE,1) and the number of nodes is stored
in ELTOP(NELE,2). The node numbers associated with the element
NELE are stored in ELTOP(NELE,i) where i=3,4,...,m+2, where
m is the number of nodes in that element. n is the total number of elements
in the mesh
The nodal freedom array NF contains the freedom numbers
associated with each node. The nodal freedom array must
contain the freedom numbers for every one of the p nodes
in the problem being considered. A fuller description
of the contents of NF can be found in th routine
document FORMNF.
The steering vector STEER is of length k, where k$=$DOFNOD*NODEL
is the total number of degrees of freedom in the element
NELE, and contains the freedom numbers associated with
element NELE in the local node order. This order is given
in the shape function document.
\end{description}
\begin{references}
None.
\end{references}
\begin{parameters}
\param{NELE} INTEGER.\\
      On entry, NELE specifies the number of the element
      for which a steering vector is to be constructed.
      Unchanged on exit.
\param{ELTOP} INTEGER array of DIMENSION (IELTOP,JELTOP) where\\
      IELTOP *GE* n and JELTOP *GE* m+2.
      Before entry, ELTOP must contain the element topologies
      element type, and the number of nodes in the element.
      Unchanged on exit.
IELTOP - INTEGER.
      On entry, IELTOP specifies the first dimension of array
      ELTOP as declared in the calling (sub)program.
      Unchanged on exit.
JELTOP - INTEGER.
      On entry, JELTOP specifies the second dimension of array
      ELTOP as declared in the calling (sub)program.
      Unchanged on exit.
\param{NF} INTEGER array of DIMENSION (INF,JNF) where INF *GE* p and\\
      JNF *GE* DOFNOD.
      Before entry, NF must contain the freedom numbers
      associated with each node. p is the total number of nodes
      in the problem.
      Unchanged on exit.
\param{INF} INTEGER.\\
      On entry, INF specifies the first dimension of array NF as
      declared in the calling (sub)program. INF *GE* p.
      Unchanged on exit.
\param{JNF} INTEGER.\\
      On entry, JNF specifies the second dimension of array NF as
      declared in the calling (sub)program. JNF *GE* DOFNOD.
      Unchanged on exit.
DOFNOD - INTEGER.
      On entry, DOFNOD specifies the number of degrees of freedom
      at each node on the element NELE.
      Unchanged on exit.
\param{STEER} INTEGER array of DIMENSION (ISTEER) where ISTEER *GE* k.\\
      On successful exit, STEER contains the freedom numbers
      associated with element NELE arranged the local node order.
      k is the total number of degrees of freedom on element NELE.
ISTEER - INTEGER.
      On entry, ISTEER specifies the dimension of the array STEER
      as declared in the calling (sub)program.
      Unchanged on exit.
\input itest.tex
\end{parameters}
\begin{indicators}
Error detected by the routine:
\error{1} On entry, NELE or DOFNOD {{<}} 0.
\error{2} On entry, IELTOP < NELE.
\error{3} On entry, JNF < DOFNOD.
\error{4} _ITEST = 4  Attempt to access element outside the
                 bounds of array ELTOP. JELTOP < m+2 where
                 m is the number of nodes in the elment NELE.
\error{5} _ITEST = 5  Attempt to access an element outside the
                 bounds of array NF. INF < p, where p is
                 the total number of nodes.
\error{6} _ITEST = 6  Attempt to access an element outside the
                 bounds of array STEER. ISTEER < k, where
                 k = DOFNOD*NODEL, the total number of
                 degrees of freedom in element NELE.
\end{indicators}
\begin{routines}
This routine uses the Level 0 Library routine ERRMES.
\end{routines}
\begin{timing}
Time taken is proportional to m and DOFNOD.
\end{timing}
\begin{storage}
There are no internally declared arrays.
\end{storage}
\begin{accuracy}
Not applicable.
\end{accuracy}
\begin{comments}
None.
\end{comments}
\end{document}
