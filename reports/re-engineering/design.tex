
\documentclass[a4paper,titlepage,11pt]{article}
\usepackage{verbatim,seg-report}%REPORT
%\usepackage{verbatim,msghtml}%HTML

\begin{document}
\author{Dr C Greenough}
\title{Re-Engineering the Finite Element Library: \\The Transformation of
 a Legacy Fortran Library}
\date{October 2003}
\refnum{RAL-TR-2004-011}
\email{c.greenough@rl.ac.uk}

%\maketitle %HTML
\abstract{%REPORT
%\begin{abstract} %HTML
The Finite Element Library ({\tt FELIB}) \cite{felib1,felib2,felib3} was 
first designed and implemented in the early 1980s and since then 
there have been four releases of the software - Release 4 is the 
current release.
\apar
Many individuals and groups have made use of {\tt FELIB} in developing
finite element based applications and in the teaching of finite
element techniques. Some 3000 known copies of the library are known
to exist (through monitoring {\it httpd} and {\it ftp} accessing) and there are
probably many other copies obtained through third parties.
\apar
This report details the re-design and re-engineering of the original Fortran~77 
{\tt FELIB} to make use of the new features of Fortran~90/95. This process provides
a very useful way in assessing some of the software tools which can assist in
this transformation and re-design. The report contains short summaries
on tool such as {\tt TOOLPACK} and {\tt plusFORT} used in this work.
\apar
The basic design~goals are discussed in light of Fortran~90/95 features and 
methods of implementation detailed. There is a short debate on whether to use
{\tt POINTER}s to arrays or {\tt ALLOCATABLE} arrays and the overall {\tt MODULE}
structure of this implementation is described.
\apar
Full details of the Fortran~77 and Fortran~90/95 versions of {\tt FELIB}
are to found on the Mathematical Software Group Web site under 
{\tt FELIB}.
}%REPORT
%\end{abstract}%HTML
\keywords{%REPORT
%{\noindent\par\bf Keywords: \rm %HTML
%
finite element, library design, Fortran 95, legacy software
}
\maketitle %REPORT
\section{Introduction}
The Finite Element Library ({\tt FELIB}) \cite{felib1,felib2,felib3} was 
first designed and implemented in the early 1980s and since then 
there have been four releases of the software - Release 4 is the 
current release.

Many individuals and groups have made use of {\tt FELIB} in developing
finite element based applications and in the teaching of finite
element techniques. Some 3000 known copies of the library are known
to exist (through monitoring \textit{httpd} and \textit{ftp} accessing) and there are
probably many other copies obtained through third parties.

The original library, based on an existing prototype library of Prof~IM~Smith of
Manchester University \cite{imsmith1}, was implemented in Fortran~66, with 
subsequent versions making use of Fortran~77, as part of one of the Rutherford 
Appleton Laboratory's Engineering Support programs funded by the then SERC 
(now EPSRC). The Numerical Algorithms Group Ltd (Oxford) provided the primary 
outlet to the scientific community. The prototype library was highly re-engineered
and its two-level structure fully documented.

Although other language versions of the base Fortran~66 library were partially 
implemented - C, DAP-Fortran and Ada - it is not until recently that is was 
thought useful to consider developing a Fortran~90/95 version. In 1998 Prof~Smith
and Dr~Griffiths \cite{imsmith2} published a Fortran~90/95 version their library which has
grown significantly through the years and there is also a version for SMPD machines \cite{imsmith3}.

Within the CLRC programme FELIB has developed and PVM/MPI versions have been developed \cite{felib4}
as too have a family of pre- and post processing routines tailored to FELIB.
Even though originally designed in the 80s {\tt FELIB} is still well used judged
by the level of Email inquiries and comments.

\section{Design goals - old and new}
\label{design}
The primary goal of {\tt FELIB} was to provide a tool box of routines components
providing the majority of the steps required in a finite element base
analysis together with a selection of \textit{example}  programs to illustrate
there use. The two-level structure of the library satisfied the requirement.

The first Finite Element Library was targeted at Fortran~66. Even
though Fortran~77 had been defined there were insufficient compilers to make
the language attractive to developers. In 1985 it was thought that \textit{
Fortran~77 was not yet universal} \cite{felib3}. Since then {\tt FELIB} has taken
on Fortran~77 and as mentioned above implementations have been started in
a number of other languages. The same could be said of Fortran~90/95 now - has
it become \textit{universal}? Again, {\it no} is probably the answer! However with Fortran~2000
nearing the completion of its definition and with the availability of many more
Fortran~90/95 compilers ({\tt g95} will may well appear soon and the Intel compiler is
available on Linux systems) it would appear that it has nearly achieved {\it universal} status.

Hence a Fortran~90/95 version of {\tt FELIB} would not appear ill-timed. What should
we be looking for in this new version? Clearly an exploitation of the new features
of Fortran~90/95: arrays, modules, dynamic memory allocation etc. The goals of the first 
library have not changed in providing a prototyping tool. However, as Fortran~90/95 has
Fortran~77 as a complete subset (apart from a list of deprecated features) it
would seem reasonable that there should be an easy migration path from
the old form of the library to the new.

To this end {\tt FELIB90} contains a variety of routines that are not strictly
necessary in Fortran~90/95. For example, the routine for initialising an
array to zero, {\tt MATNUL} and the matrix multiplication routine
{\tt MATMUL}. As explained below these can be replaced by the use of
some of the basic features of Fortran~90/95 or by the use of standard
intrinsic functions.

Fortran~90/95 provides mechanisms for defining application specific data
types and structures. The simple array definitions of arrays such as the
mesh geometry, {\tt COORD}, could be replace by more object oriented
structures. However it was decide to retain the simple array structures
for the main data elements of the library.

Also {\tt FELIB} used a large number of intermediate arrays such {\tt DTPD} and
{\tt ELK}. These aided the readability of the {\tt FELIB} programs and allowed
them to mirror the mathematical analysis. Fortran~90/95 provides a number
of mechanisms for allocating array space dynamically. In this re-design we
will use these features to reduce the number of array definitions, space
allocations and initialisations the programmer is require to perform. 

For example the routine {\tt QQUA4} returns the abscissas and weights of a
four-point quadrature rule. Its {\tt FELIB} call is
\begin{quote}
\tt\source
CALL QQUA4(WGHT,IWGHT,ABSS,IABSS,JABSS,NQP,ITEST)
\end{quote}
The arrays {\tt WGHT} and {\tt ABSS} are intermediate arrays whose sizes are
determined by the content of the subroutine. They are only required for
the assembly phase and the memory they occupy could be released once this
phase is completed. By making these arrays dynamic and allowing the routine
to allocate appropriate space the programmer task can be simplified. The call to the
{\tt FELIB90} version of the routine could be reduced to
\begin{quote}
\tt\source
CALL QQUA4(wght,abss,nqp)
\end{quote}
where the arrays {\tt wght} and {\tt abss} would be returned with the correct
size, populated with data together with {\tt nqp}. The user would still be
required to provide a declaration of the arguments types but not necessarily
their sizes. The routine would manage the allocation of memory space and 
initialisation.

As with many programs the initial lines of code are given over to declarations
of variables and initialisations. In {\tt FELIB} these could take up 10 to 30
lines. To aid this declaration process each {\tt FELIB90} example program will 
be provided with a definitions file which will specify the types of the most 
common variable and intermediate arrays. 
An example of one of these is given in Section~\ref{prog-def}.

\section{Use of Fortran~90/95 features}
Fortran~90/95 has made many additions to improve the Fortran language and some that
open up new approaches to library design and implementation. In this section
we will highlight some of these changes and consider how they might improve
the design of a Fortran~90/95 version of the Finite Element Library.
\subsection{Array features}
The array features are one clear example of an important Fortran~90/95 feature. It is now
possible to initialise and perform calculations on complete arrays whereas
before, in Fortran~77, this required multiple nested loop structures.

To initialise the array {\tt SYSK} to zero in Fortran~77 required:

\begin{quote}
\source	
\begin{verbatim}
   INTEGER ISYSK, JSYSK
   PARAMETER (ISYSK=200, JSYSK=20)
   REAL SYSK(ISYSK,JSYSK)
\end{verbatim}
\ldots\ldots
\begin{verbatim}
   DO 10 I=1,TOTDOF
      DO 20 J=1,HBAND
         SYSK(I,J) = 0.0D0
20    CONTINUE
10 CONTINUE
\end{verbatim}
\end{quote}
or
\begin{quote}
\source
\begin{verbatim}
   CALL MATNUL(SYSK,ISYSK,JSYSK,TOTDOF,HBAND,ITEST)
\end{verbatim}
\end{quote}
using the {\tt FELIB} routine. Other often used routines during the element
matrix construction and assembly are {\tt MATADD}, {\tt MATRAN} and {\tt MATMUL}. 
The array features of Fortran~90/95 provide a more compact and natural way of performing
these actions. For example:
\begin{quote}
\tt\source
SYSK=0
\end{quote}
can replace a call to {\tt MATNUL}.
\begin{quote}
\tt\source
ELK = ELK + DTBD
\end{quote}
would replace a call to {\tt MATADD}.

Fortran~90/95 provides a number of basic array/matrix intrinsic functions. In the
context of {\tt FELIB} three of the more important are {\tt MATMUL}, {\tt TRANSPOSE}
 and {\tt DOT\_PRODUCT}. So
\begin{quote}
\tt\source
{~}CALL MATMUL(LDER,ILDER,JLDER,GEOM,IGEOM,JGEOM,\\
{*~~~~~~~~~~~~}JAC,IJAC,JJAC,DIMEN,NODEL,DIMEN,ITEST)
\end{quote}
can be replaced by the considerable simpler
\begin{quote}
\tt\source
JAC = MATMUL(LDER,GEOM)
\end{quote}
and
\begin{quote}
\tt\source
CALL SCAPRD(GEOM(1,1),IGEOM,FUN,IFUN,NODE,X,ITEST)
\end{quote}
replace by
\begin{quote}
\tt\source
X = DOT\_PRODUCT(GEOM(1:NODEL,1),FUN)
\end{quote}
Immediately one notices the reduction in actual arguments in the routines calls.
Fortran~90/95 carries much more information about arrays than Fortran~77 did. Properties
as an array's {\it rank} or an array's {\it shape} are readily available through
a range of {\tt INTRINSIC} functions. For example {\tt RANK}, {\tt SHAPE}, {\tt SIZE},
{\tt LBOUND} and {\tt UBOUND}. These, together with the stricter conditions on the
property matching between actual and dummy arguments in procedure calls, enable a
considerable amount of information on arrays to be passed implicitly into procedures.
Although these new features can simplify or modify many of the steps involved in
an {\tt FELIB} program there is a possible danger in compactness making the
software more opaque. For example the construction of one quadrature point contribution
to an element stiffness matrix could be written as:
\begin{quote}
\tt\source
ELK = ELK + MATMUL(TRANSPOSE(MATMUL(jacin,lder)),   \&\\
\hspace*{30mm}MATMUL(p,MATMUL(jacin,lder)))
\end{quote}
Clearly this is more compact but it is hardly transparent. A balance will be required
between succinctness and clarity particularly in software that is to be used
as a teaching aid.
\subsection{Dynamic storage allocation}
An important feature of Fortran~90/59 is the ability to
dynamically allocate storage to arrays at execution time. Fortran~90/95
provides two mechanisms to make this possible: the {\tt ALLOCATABLE} array
and the {\tt POINTER} array. Both these types can be specified without
size information:

\begin{quote}
\tt\source
REAL (wp), ALLOCATABLE :: sysk(:,:)\\
REAL (wp), POINTER :: sysm(:,:)
\end{quote}

These statements define the two arrays {\tt sysk} and {\tt sysm} whose memory
allocation can be specified thus:

\begin{quote}
\tt\source
ALLOCATE(sysk(100,10))\\
ALLOCATE(sysm(100,10))
\end{quote}
In the first of these {\tt sysk} has been defined as an {\tt ALLOCATABLE} array and
once memory is allocate {\tt sysk} is a unique reference to this memory. The second
allocate creates a similar reference to memory which can be used in the same way.
However this may not be a unique reference. The nature of {\tt POINTER}s allow
multiple references to the same locations in memory. Although this would not be
intended in the context of {\tt FELIB} it is a possibility.

This is one of the features of {\tt POINTER} arrays that the programmer must be
aware. This non-uniqueness inherent in {\tt POINTER}s can lead to unfortunate
side effects.

Probably the most important of these is the possibility of {\it memory leaks} if the 
arrays are not allocated and deallocated assiduously. It is very easy to allocate space  
to a {\tt POINTER} and then to later re-allocate a different section of memory to the 
same {\tt POINTER} without de-allocating the former. In general compilers and run-time 
system will not flag this as an error. 
However the result is that there will be sections of memory reserved
with no way of referencing it. This would have the potential of {\it eating} away
the memory available to the application if the {\tt ALLOCATE} is contained in some form
of loop. 

A second characteristic of {\tt POINTER}s that can be a nuisance is that of initialisation. 
The {\tt POINTER} declaration above defines a {\tt POINTER} to an array but does not give 
an initial value to the {\tt POINTER}. As a consequence the result from the intrinsic 
function {\tt ASSOCIATED} is really undefined although, as is often the case, many compilers
set these variables automatically to {\tt NULL}. However, as this is the only mechanism 
through which it is possible to determine whether a {\tt POINTER} has been associated 
with a target, it is essential that {\tt POINTER} variables be explicitly initialised. 
This can be readily achieved using either the {\tt NULL()} intrinsic or the {\tt NULLIFY} 
statement. For example:
\begin{quote}
\tt\source
REAL (wp), POINTER :: sysm(:,:) => NULL()
\end{quote}
will define and initialised to {\tt NULL} the {\tt POINTER sysm} whereas
\begin{quote}
\tt\source
NULLIFY(sysm)
\end{quote}
performs the same task as an executable statement.

An important point to consider in the library context is how these array types can
be passed into subroutines and functions and how their dynamic properties can
be exploited. In the Fortran~90/95 standard it was not possible to pass {\tt ALLOCATABLE}
arrays as dummy arguments. Passing {\tt POINTER} arrays was allowed. However, during
2001 an extension to allow the passing of {\tt ALLOCATABLE} arrays into procedures
was proposed \cite{cohen}. This was excepted by the Fortran Standards body but
as of yet only a few Fortran~90/95 compilers support for this feature. In a series
of simple checks on the passing of {\tt ALLOCATABLE} arrays into procedures using
a number of compilers (Intel {\tt ifc}, Lahey {\tt lf95}, {\tt NAG f95}, {\tt DEC f95}):
one gave an error message (ifc), one gave a warning and continued compilation ({\tt NAG f95}), 
two compiled without warnings ({\tt lf95}, {\tt DEC f95}). Only two the those that compiled and
linked successfully executed correctly (Lehay {\tt lf95}, {\tt NAG F95}).

Not supporting this feature stops the programmer passing an unallocated
{\tt ALLOCATABLE} array into procedure and making the allocation within 
the procedure. Defining arrays as {\tt ALLOCATABLE} is one method of
providing dynamic arrays within Fortran~90/95. However, if the arrays within {\tt FELIB90} 
are defined as {\tt ALLOCATABLE} to provide this functionality it will inhibit 
dynamically allocated intermediate arrays as described above
in Section~\ref{design} as desirable. To overcome this
problem all the arrays within {\tt FELIB90} will be of a {\tt POINTER} type. Although
this may lead to problems of memory leaks, by the use of a stack of dynamic arrays
and a type of semi-automatic garbage collection, these potential problems can
be minimised from the library's point of view. However it will still be
possible for the user to fall foul of these difficulties.

\subsection{Optional arguments}
Another important new feature of Fortran~90/95 is the provision of optional dummy arguments
in procedure calls. These can shorten the argument lists of library routines
significantly. A simple example of the use of optional arguments is the
{\tt PRTGEO} routine. This routine outputs the coordinate array {\tt COORD}. The
full argument list would be:
\begin{quote}
\tt\source
CALL PRTGEO(coord,totnod,dimen,nout,itest)
\end{quote}
This can be reduce to
\begin{quote}
\tt\source
CALL PRTGEO(coord)
\end{quote}
if the size and shape of {\tt coord} matches exactly the element coordinate data and 
there is a default output channel {\tt nout}. {\tt totnod} and {\tt dimen} can be
determined using the intrinsic function {\tt SIZE}.

A design goal of {\tt FELIB90} has been to reduce required arguments lists to a minimum
whilst maintaining flexibility and control through optional arguments.

Because of the nature and implementation of optional arguments in Fortran~90/95 the
head of most routines has a group of code dealing with optional arguments. For an
optional argument {\tt arg1} it takes the form
\begin{quote}
\source
\begin{verbatim}
! Dummy argument
INTEGER, OPTIONAL :: arg1
.......
! Local variables
INTEGER :: larg1
.......
IF (PRESENT(arg1))THEN
larg1 = arg1
ELSE
larg1 = default_arg1
ENDIF
\end{verbatim}
\end{quote}
where {\tt larg1} is the local variable associated with {\tt arg1}. It should
be note that {\tt arg1} cannot be referenced if it is not present. It may only
be reference through the {\tt PRESENT} intrinsic.

In general {\tt FELIB90} only has scalars as optional arguments. 
To aid clarity {\tt FELIB90} uses a simple set generic routines to assign default values to
optional arguments. These are provided through the module {\tt mod\_setopt}. The part
of the code is shown below.
\begin{quote}
\source
\begin{verbatim}
MODULE mod_setopt

USE FELIB_GLOBALS, only : wp

INTERFACE setopt
   MODULE PROCEDURE complex_setopt
   MODULE PROCEDURE real_setopt
   MODULE PROCEDURE integer_setopt
   MODULE PROCEDURE character_setopt
END INTERFACE

CONTAINS

...........................

SUBROUTINE real_setopt(dummy,value,actual)
IMPLICIT NONE
REAL (wp) :: dummy, value
REAL (wp) , OPTIONAL :: actual

IF( PRESENT(actual)) THEN
  dummy=actual
ELSE
  dummy=value
ENDIF

END SUBROUTINE real_setopt

SUBROUTINE integer_setopt(dummy,value,actual)
IMPLICIT NONE
INTEGER :: dummy, value
INTEGER , OPTIONAL :: actual


IF( PRESENT(actual)) THEN
  dummy=actual
ELSE
  dummy=value
ENDIF

................................

END MODULE mod_setopt
\end{verbatim}
\end{quote}
Now the setting of optional arguments is reduced to:
\begin{quote}
\source
\begin{verbatim}
call setop(larg1,default_arg1,arg1)
\end{verbatim}
\end{quote}
Use of this routine clears up the opening statements of the library routines.
%
\subsection{Generic routines}
As can be seen from the above another very useful feature of Fortran~90/95 is the definition 
of {\it generic} interfaces: functions with differing argument types being called by
the same {\it generic} name. This an example of the idea of 
{\it overloading} in Fortran~90/95. {\tt FELIB90} uses generic routines where ever thought useful.
Routines such as {\tt MATNUL} can be made generic and thus capable of operating on
{\tt INTEGER}, {\tt REAL} and {\tt COMPLEX} arrays. As seen above {\tt setopt} is define as a generic
routine capable of operating on {\tt INTEGER}, {\tt REAL}, {\tt COMPLEX} and {\tt CHARACTER} variables. 

In the Fortran~66 and Fortran~77 versions of the library {\tt COMPLEX} variables were
treated as order pairs of {\tt REAL} values and required their own set of manipulation
routines such as {\tt CMTNUL} and {\tt CSYSOL}. Fortran~90/95 defines a {\tt COMPLEX}
data type which much simplifies some of the routines through the use of {\it generic}
interfaces. As a consequence this does require that many routines are defined as generic routines.
%
\subsection{Intrinsic functions}
Where ever possible the standard Fortran~90/95 {\tt INTRINSIC} functions have
been used in the Level~0 Library routines. For example {\tt HUGE} is used to
obtain the largest {\tt REAL} and largest {\tt INTEGER} available to the library.
{\tt EPSILON} is use to determine the smallest value for which $1+\epsilon>1$.
There are one or two problems with using the standard Fortran~90/95 intrinsics. One of the design goal of
{\tt FELIB90} to provide an easy migration path requires that potentially obsolete routines
be provide. {\tt MATMUL} is probably the most important of these. The name {\tt MATMUL}
conflicts with the Fortran~90/95 intrinsic of the same name and will cause compiler 
warnings or errors if {\tt MATMUL} is either declared or used in this way.

To circumnavigate this problem a new interface to the Fortran~90/95 intrinsic
has been defined. This is another feature of Fortran~90/95: procedures defined
in modules can have their names aliased. 
{\tt FELIB90} defines an alias to the intrinsic {\tt MATNUL} as {\tt MAXTRIX\_MULTIPLY}:
 (similar to {\tt DOT\_PRODUCT} and {\tt TRANSPOSE} in style).
This is defined in the simple module
\begin{quote}
\source
\begin{verbatim}

    MODULE mod_matmul_intrinsic
      
       INTRINSIC MATMUL

    END MODULE mod_matmul_intrinsic
\end{verbatim}
\end{quote}
and a {\tt USE} statement at the head of the {\tt FELIB90} module.
\begin{quote}
\tt\source
USE mod\_matmul\_intrinsic, ONLY : matrix\_multiply => matmul
\end{quote}
These statements essentially map the name {\tt matrix\_multiply} onto
the intrinsic function {\tt MATMUL}. Through this mechanism the intrinsic
{\tt MATMUL} is made available to {\tt FELIB90} programs. Thus
\begin{quote}
\source
\tt
CALL matmul(lder,geom,jac)
\end{quote}
and
\begin{quote}
\source
\tt
jac=MATRIX\_MULTIPLY(lder,geom)
\end{quote}
are equivalent.
%
\section{Management of dynamic arrays}
As mentioned above it was decided to implement the library
using {\tt POINTER} arrays to provide the maximum flexibility
in managing intermediate and temporary arrays even though this does have some potential
pit falls. However these can be minimised by providing the user with a collection
of array management routines although it is not really expected that the
user would make use of these routines. It is thought that as more compilers adopt the 
recommendations of the ISO~TR~15581 this approach can be easily modified to make
use of {\tt ALLOCATABLE} arrays.

In this version of {\tt FELIB90} a module ({\tt mod\_space}) of memory management 
routines based on {\tt POINTER} arrays has been implemented. The module
contains three types of routine: one set the {\tt create} vectors and arrays,
a second set to {\tt destroy} and another the {\tt check} the memory
allocations.

The module {\tt mod\_space} maintains and manages a set of arrays which point to 
specific arrays and vectors. The basic allocation process starts with a call to 
{\tt create} within a subroutine.
\begin{quote}
\source
\tt
CALL create(elk,dofel,dofel)
\end{quote}
The {\tt create} routine performs the following steps:
\begin{quote}
\it
\begin{tabbing}
-----\=-----\=-----\=-----\=-----\=-----\=-----\=-----\kill
{\bf loop}~list of current allocations for an association\\ 
\>to {\tt elk} using the {\tt associated} intrinsic. \\
\>{\bf if} a current association exists {\bf then}\\
\>\>{\bf if}~size and shape of allocated space is ok {\bf then}\\
\>\>\> return to calling routine\\
\>\>{\bf else}\\
\>\>\> deallocate using {\tt destroy}\\
\>\>\> create new space using {\tt create} (recursively)\\
\>\>{\bf endif}\\ 
\>{\bf else}\\
\>\>find next free pointer in list\\
\>\>allocate new space\\
\>\>associate with target name\\
\>\>mark pointer as in use\\
\>{\bf endif}\\
{\bf end loop}
\end{tabbing}
\end{quote}
{\tt create} is a generic routine and provides for vectors and two-dimensional 
arrays of the basic types required by {\tt FELIB90}: {\tt real, integer} and  {\tt complex}.
The module {\tt mod\_space} also provides: {\tt destroy}, a generic routine for
deallocating memory space and disassociating pointers and targets and {\tt check},
a routine to check on the status of a vector or array.
%
\section{Library construction}
{\tt FELIB90} has taken on a modular approach to its design - the library is
a {\tt MODULE} to be {\tt USE}d by the user program. All the {\tt FELIB90}
routines are contained in their own modules and these are {\tt USE}d
by the {\tt FELIB90} module to build the complete library. A consequence of
this is that all the interfaces of the library routines are explicit and
can be used by the compiler at compile and run time to provide diagnostics.

This approach has some impact on developing the library in terms of
compilation but these are minimal as {\tt FELIB90} is a \textit{small} library. There
are benefits in as much that the developer is not required to generate
interface blocks for the user to reference.

As mentioned above each routine of {\tt FELIB90} is a module in its own right. This
makes providing generic interfaces straightforward and ensures explicit
interfaces. So for example the routine {\tt MATNUL} has the following code
\begin{quote}
\source
\begin{verbatim}
    MODULE mod_matnul

!-----------------------------------------------------------------------
! PURPOSE
!      MATNUL creates and sets matrix A to the null matrix

! HISTORY

!      Copyright (C) 2000 : CCLRC, Rutherford Appleton Laboratory
!                           Chilton, Didcot, Oxfordshire OX11 0QX

!      Release 1.0   2 Jul 2000 (CG)

! ARGUMENTS in
!      M       number of rows of A set to zero (OPTIONAL)
!      N       number of columns of A set to zero (OPTIONAL)
!      ITEST   error checking option (OPTIONAL)

! ARGUMENTS out
!      A       array set to zeros

! ROUTINES called

!***********************************************************************

      USE felib_globals,only : wp
      USE mod_space,only : create
      USE mod_setopt,only : setopt
      
      PRIVATE
      PUBLIC matnul

      INTERFACE matnul
        MODULE PROCEDURE complex_matnul
        MODULE PROCEDURE real_matnul
        MODULE PROCEDURE integer_matnul
      END INTERFACE

      CHARACTER (5) :: srname = 'MATNUL'    

    CONTAINS

..............................
     
      SUBROUTINE real_matnul(a,m,n,itest)
        IMPLICIT NONE

! Dummy arguments

        INTEGER, OPTIONAL :: m, n, itest
        REAL (wp), POINTER :: a(:,:)
        INTENT (IN) :: m,n
        INTENT (INOUT) :: itest

! Local variables

        INTEGER :: i, ierror, j, mm, nn
        CHARACTER (5) :: srname = 'MATNUL'

! Check optional arguments and association of A

        CALL setopt(mm,size(a,1),m)
        CALL setopt(nn,size(a,2),n)

! Create A if necessary then initialise

        CALL create(a,mm,nn)

        a(1:mm,1:nn)=0

      END SUBROUTINE real_matnul

..............................

    END MODULE mod_matnul
\end{verbatim}
\end{quote}
As can been seen from the source {\tt MATNUL} is a generic routine that will if
necessary create the storage to be associated with a variable using the routine
{\tt CREATE} and manages optional arguments using the {\tt SETOPT} routine.
This is type of {\tt FELIB90} routines.

The final step is to build the full library, {\tt FELIB90}. Again {\tt FELIB90} is a module
in its own right and contains a sequence of {\tt USE} statements to include
all the library routines - one per routine. Below is show the {\tt FELIB90} code.
\begin{quote}
\source
\begin{verbatim}
    MODULE felib90

! MODULE felib90 is the main defining modules of FELIB90. All
! user callable routines are included here.

! System

      USE felib_globals, ONLY : wp
      USE mod_space, ONLY : create, destory, check

! Routines

      USE mod_bndwth, ONLY : bndwth
      USE mod_qqua4, ONLY : qqua4
      USE mod_elgeom, ONLY : elgeom
      USE mod_quam4, ONLY : quam4
      USE mod_quam8, ONLY : quam8
      USE mod_scaprd, ONLY : scaprd
      USE mod_matmul, ONLY : matmul
      USE mod_matnul, ONLY : matnul
      USE mod_matran, ONLY : matran
      USE mod_matvec, ONLY : matvec
      USE mod_prtval, ONLY : prtval
      USE mod_asrhs, ONLY : asrhs
      USE mod_assym, ONLY : assym
      
      .............................
      
      USE mod_chosol, ONLY : chosol
      USE mod_direct, ONLY : direct
      USE mod_matinv, ONLY : matinv
      USE mod_getgeo, ONLY : getgeo
      USE mod_gettop, ONLY : gettop
      USE mod_prtgeo, ONLY : prtgeo
      USE mod_prttop, ONLY : prttop
      USE mod_errmes, ONLY : errmes
      USE mod_asful, ONLY : asful
      USE mod_vecnul, ONLY : vecnul
      USE mod_vecadd, ONLY : vecadd
      USE mod_matadd, ONLY : matadd
      USE mod_setopt, ONLY : setopt

! Redefinition of intrinsic MATMUL

      USE mod_matmul_intrinsic, ONLY : matrix_multiply => matmul

    END MODULE felib90
\end{verbatim}
\end{quote}
\section{Program definitions}
\label{prog-def}
To help make the programs more readable the definitions of many of the
standard intermediate arrays such as {\tt ELK} and {\tt GEOM} have been
collected together into a definitions module - for example {\tt def3p1}.
These are provided with the program file and are to be compiled with
the user program. Clearly the user can add to these definitions if thought
useful or replace them with their own specific definitions. An example of
this type of module is given below.
\begin{quote}
\source
\begin{verbatim}
    MODULE def3p1

! All arrays within FELIB programs are defined as POINTERs. This
! enables dynamic allocation within FUNCTIONS and SUBROUTINES. Many
! routines automatically ALLOCATE space for a current set of intermediate
! variables use in the solution process.

! These variables are created through a library of space management routines
! which includes garbage collection. These routines do not inhibit the user
! defining his own arrays locally or by using the space creataion routines.

! This modules contains a standard set of variable definitions
! often found in basic FELIB programs.

      USE felib_globals, ONLY : wp


! Allocatable arrays - dependent on element types and problem 
!                      dimensionality

      INTEGER, POINTER ::     &
        steer(:)   => null()    ! Element steering vector
      REAL (wp), POINTER ::   &
        elq(:)     => null(), & ! Element vector                              
        fun(:)     => null(), & ! Shape function vector
        xy(:)      => null(), & ! Global coordinate vector
        geom(:,:)  => null(), & ! Local geometry array
        elk(:,:)   => null(), & ! Element stiffness arrat
        lder(:,:)  => null(), & ! Local derivatives of shape funtions
        jac(:,:)   => null(), & ! Transformation jacobian
        jacin(:,:) => null(), & ! Inverse of JAC
        geomt(:,:) => null(), & ! Element geometry transposed
        wght(:)    => null(), & ! Quadrature weights
        abss(:,:)  => null(), & ! Quadrature abssise
        p(:,:)     => null(), & ! Permeitvity array P
        pd(:,:)    => null(), & ! P transposed
        scvec(:)   => null(), & ! Element source vector
        gder(:,:)  => null(), & ! Global derivatives of shape functions
        dtpd(:,:)  => null(), & ! Element matrix
        gdert(:,:) => null()    ! Tranpose of GDERT


    END MODULE def3p1
\end{verbatim}
\end{quote}

\section{An example program}
In this section a complete {\tt FELIB90} Level~1 program is shown. The structure is
very much like that of the Fortran~77 programs. When compared with the Fortran~77
version it can be seen the actual argument lists of the routines are much shorter. 
A comparison with the Fortran~77 program {\tt  SEG3P1} \cite{felib1} will also show
the similarity of structure and therefore this program should be recognisable to 
existing users of {\tt FELIB}.

Throughout this program the shortest possible argument list have been used and use has
been made of all the system defaults provided by the library. In particular
input/output channel numbers.

As mention above {\tt FELIB90} contains some redundant routines and code to aid
the transition from {\tt FELIB} to {\tt FELIB90} and also to aid clarity in a teaching
context. For the programmer who wishes to move to a full Fortran~90/95
implementation a number of \textit{Notes} have been added to each
Level~1 program. The next section, Section~\ref{notes}, give the notes on this program.

\begin{quote}
\source
 \begin{verbatim}
     !*****************************************************************
     
1     PROGRAM seg3p1
     
     !*****************************************************************
     
     !     Copyright (C) 2003 : CLRC, Rutherford Appleton Laboratory
     !     Chilton, Didcot, Oxfordshire OX11 0QX
     
     ! N.B. The working precision of the current library is held
     !      in the variable WP. This must be used in all REAL
     !      declarations of variables used by FELIB90.
     
     !      The program also uses the standard FELIB90 values for 
     !      nin and nout.
     
     !*****************************************************************
     
2       USE felib90 ! Use FELIB90 all routines
3       USE def3p1 ! Use standard SEG3P1 definitions
4       IMPLICIT NONE
     
     ! Parameters
     
5       REAL (WP), PARAMETER :: scale = 1.0E+10
     
     ! Local variables
     
6       INTEGER :: bndnod, dimen, dofel, dofnod, hband, i, iquad, itest, j, &
7         nele, nodel, nqp, totdof, totels, totnod
8       REAL (WP) :: det, eta, quot, strgth, x, xi, y
     
     ! Allocatable arrays - mesh size dependent - user defined in the data
     
9       INTEGER, POINTER :: bnode(:), nf(:,:), eltop(:,:)
10      REAL (WP), POINTER :: bval(:), rhs(:), coord(:,:), sysk(:,:)
     
     ! Intrinsic functions
     
11      INTRINSIC abs
     
     !     Initialisation of POINTERS to main arrays
     
12      NULLIFY (bnode,nf,eltop)
13      NULLIFY (bval,rhs,coord,sysk)
     
     !     Set error checking flag
     
14      itest = 0
     
     !     **********************
     !     *                    *
     !     * Input Data Section *
     !     *                    *
     !     **********************
     
     !     Input of nodal geometry
     
15      CALL getgeo(coord,totnod,dimen)
16      CALL prtgeo(coord)
     
     !     Input of element topology
     
17      CALL gettop(eltop,totels)
18      CALL prttop(eltop)
     
     !     Input of permeabilities, construction of permeability matrix P
     !     and source strength
     
19      CALL matnul(p,dimen,dimen)
20      WRITE (nout,'(/A)') 'Permeabilities'
21      READ (nin,'(2F10.0)') (p(i,i),i=1,dimen)
22      WRITE (nout,'(2F10.5)') (p(i,i),i=1,dimen)
     
23      WRITE (nout,'(/A)') 'Source Strength'
24      READ (nin,'(F10.0)') strgth
25      WRITE (nout,'(F10.5)') strgth
     
     !     Input of number of degrees of freedom per node, input of
     !     boundary conditions and construction of nodal freedom array NF
     
26      WRITE (nout,'(/A)') 'Degrees of freedom per node (DOFNOD)'
27      READ (nin,'(I5)') dofnod
28      WRITE (nout,'(I5)') dofnod
     
     !    Input boundary conidtions
     
29      WRITE (nout,'(/A)') 'Boundary Conditions'
30      READ (nin,'(I5)') bndnod
31      WRITE (nout,'(I5)') bndnod
     
32      CALL vecnul(bnode,bndnod)
33      CALL vecnul(bval,bndnod)
34      DO i = 1, bndnod
35        READ (nin,'(I5,F10.0)') bnode(i), bval(i)
36        WRITE (nout,'(I5,F10.5)') bnode(i), bval(i)
37      END DO
     
     ! Setup nodel freedom array
     
38      CALL matnul(nf,totnod,dofnod)
39      totdof = 0
40      DO i = 1, totnod
41        DO j = 1, dofnod
42          totdof = totdof + 1
43          nf(i,j) = totdof
44        END DO
45      END DO
     
     !     Calculation of semi-bandwidth
     
46      CALL bndwth(eltop,nf,hband)
     
     !  ************************************
     !  *                                  *
     !  * System Stiffness Matrix Assembly *
     !  *                                  *
     !  ************************************
     
     
     ! System matrices setup and initalise : rhs, sysk
     
47      CALL matnul(sysk,totdof,hband)
48      CALL vecnul(rhs,totdof) 
     
     ! Setup quadrature
     
49      CALL qqua4(wght,abss,nqp)
     
     ! Begin main element loop
     
50      DO nele = 1, totels !Loop over all elements
     
51        nodel = eltop(nele,2)
52        dofel = dofnod*nodel
53        CALL elgeom(nele,eltop,coord,geom)
     
     !     Integration loop for element stiffness using NQP quadrature
     !     points
     
54        CALL matnul(elk,dofel,dofel) 
55        CALL vecnul(elq,dofel)
56        CALL vecnul(scvec,dofel)
     
57        DO iquad = 1, nqp !Numerical integration
     
     !     Form linear shape function and space derivatives in the local
     !     corrdinates. Transform local derivatives to global coordinate
     !     system
     
58          xi = abss(1,iquad)
59          eta = abss(2,iquad)
60          CALL quam4(fun,lder,xi,eta)
     
61          CALL matran(geom,geomt)
62          CALL matvec(geomt,fun,xy) 
63          x = xy(1)
64          y = xy(2)
     
65          CALL matmul(lder,geom,jac) 
66          CALL matinv(jac,jacin,det)
67          CALL matmul(jacin,lder,gder)
     
     !     Formation of element stiffness ELK
     
68          CALL matmul(p,gder,pd)
69          CALL matran(gder,gdert)
70          CALL matmul(gdert,pd,dtpd)
     
71          quot = abs(det)*wght(iquad)
72          dtpd = dtpd*quot
73          scvec = fun*src(x,y,strgth)*quot
     
74          CALL matadd(elk,dtpd)
75          CALL vecadd(elq,scvec)
     
76        END DO !Loop over quadrature points - iquad
     
     !     Assembly of system stiffness matrix
     
77        CALL direct(nele,eltop,nf,steer)
78        CALL assym(sysk,elk,steer)
79        CALL asrhs(rhs,elq,steer)
     
80      END DO !Loop over elements - nele
     
     !     *********************
     !     *                   *
     !     * Equation Solution *
     !     *                   *
     !     *********************
     
     !     Modification of stiffness matrix and right-hand side to
     !     implement boundary conditions
     
81      DO i = 1, bndnod
82        j = bnode(i)
83        sysk(j,hband) = sysk(j,hband)*scale
84        rhs(j) = sysk(j,hband)*bval(i)
85      END DO
     
     !     Solution of system matrix for the nodal values of the
     !     potential
     
86      CALL chosol(sysk,rhs)
     
87      WRITE (nout,'(/A)') 'Nodal Potentials'
88      CALL prtval(rhs,nf)
     
89      STOP
     
90    CONTAINS
     
     !*****************************************************************
     ! Source function
     
91      FUNCTION src(x,y,strgth)
     
92        USE felib90
     
93        IMPLICIT NONE
     
     ! Dummy arguments
     
94        REAL (wp) :: src
95        REAL (wp) :: strgth, x, y
96        INTENT (IN) strgth, x, y
     
97        src = 0.0D0
98        IF ((x>1.0D0) .AND. (x<2.0D0) .AND. (y>1.0D0) .AND. (y<2.0D0)) &
99          src = strgth
     
100      END FUNCTION src
     
101    END PROGRAM seg3p1
\end{verbatim}
\end{quote}

\section{An example of program \textit{notes}}
\label{notes}
% Notes on seg3p1 Fortran 90
For each Level~1 Programs  a set of {\it Notes} has been developed. 
These indicate how a programmer could modify the Level~1 Programs to use 
more fully the features of Fortran~90/95. These include the use of
explicit {\tt ALLOCATE} statements for memory allocations and the use
of other Fortran~90/95 intrinsics.
\begin{description}
%
\item[Statement 15:] The routine {\tt getgeo} allocates memory for {\tt coord} depending
on the data. The total number of nodes, {\tt totnod} and the dimensionality
of the problem, {\tt dimen} are returned. 
The array {\tt coord} can be defined and initailised using the statements 
\begin{quote}
\source
ALLOCATE(coord(totnod,dimen))\\
coord=0.0
\end{quote}
provided {\tt totnod} and {\tt dimen} are known.
%
\item[Statement 17:] The routine {\tt gettop} allocates memory for {\tt eltop}
depending on the data. The total number of elements, {\tt totels} is returned.
The array {\tt eltop} can be defined and initialised using the statements
\begin{quote}
\source
ALLOCATE(eltop(totels,max\_nodel+2))\\
eltop=0
\end{quote}
provided {\tt totels} and {\tt max\_nodel} are known. {\tt max\_nodel} is set to
the largest number of nodes in an element ({\tt nodel}) for the given mesh. For
simple meshes this will be equal to {\tt nodel}.
%
\item[Statement 19:] The routine {\tt matnul} allocates memory and initialises
the array {\tt p}. This can be performed using the statements
\begin{quote}
\source
ALLOCATE(p(dimen,dimen))\\
p=0.0
\end{quote}
\item[Statement 32:] The routine {\tt vecnul} allocates memory and initialises
the array {\tt bnode}. This can be performed using the statements
\begin{quote}
\source
ALLOCATE(bnode(bndnod))\\
bonde=0
\end{quote}
\item[Statement 33:]  The routine {\tt vecnul} allocates memory and initialises
the array {\tt bval}. This can be performed using the statements
\begin{quote}
\source
ALLOCATE(bval(bnnod))\\
bval=0.0
\end{quote}
\item[Statement 38:]  The routine {\tt matnul} allocates memory and initialises
the array {\tt nf}. This can be performed using the statements
\begin{quote}
\source
ALLOCATE(nf(totnod,dofnod))\\
nf=0
\end{quote}
\item[Statement 47:]  The routine {\tt matnul} allocates memory and initialises
the array {\tt sysk}. This can be performed using the statements
\begin{quote}
\source
ALLOCATE(sysk(totdof,hband))\\
sysk=0.0
\end{quote}
\item[Statement 48:] The routine {\tt vecnul} allocates memory and initialises
the array {\tt rhs}. This can be performed using the statements
\begin{quote}
\source
ALLOCATE(rhs(totdof)\\
rhs=0.0
\end{quote}
\item[Statement 54:] The routine {\tt matnul} allocates memory and initialises
the array {\tt elk}. This can be performed using the statements
\begin{quote}
\source
ALLOCATE(elk(dofel,dofel))\\
elk=0.0
\end{quote}
\item[Statement 55:] The routine {\tt vecnul} allocates memory and initialises
the array {\tt elq}. This can be performed using the statements
\begin{quote}
\source
ALLOCATE(elq(dofel))\\
elq = 0.0
\end{quote}
\item[Statement 56:] The routine {\tt vecnul} allocates memory and initialises
the array {\tt scvec}. This can be performed using the statements
\begin{quote}
\source
ALLOCATE(scvec(dofel))\\
scvec = 0.0
\end{quote}
\item[Statement 61 to 62 :] The routine {\tt matran} transposes the array {\tt geom}.
It creates and initialises memory for the intermediate
array {\tt geomt}. This can be performed using the intrinsic {\tt TRANSPOSE}:
\begin{quote}
\source
geomt=TRANSPOSE(geom)
\end{quote}
provided {\tt geomt} has been created and is of a suitable size and shape.

The routine {\tt matvec} post multiplies the matrix {\tt geomt}
by the vector {\tt fun}. This can be performed using the intrinsic {\tt MATMUL} 
which been mapped onto the function  {\tt MATRIX\_MULTIPLY} in FELIB90. {\tt matvec}
will create and initial memory for the intermediate array {\tt xy} so {\tt xy}
will need to be created.
\begin{quote}
\source
xy=MATRIX\_MULTIPLY(geomt,fun)
\end{quote}
\item[Statements 61 to 64:] An alternative to these statements is to 
calculate {\tt x} and {\tt y} directly through an array section 
using either the FELIB90 routine {\tt scaprd}
\begin{quote}
\source
CALL scaprd(geom(1:nodel,1),fun,x)\\
CALL scaprd(geom(1:nodel,2),fun,y)
\end{quote}
or the intrinsic {\tt DOT\_PRODUCT}:
\begin{quote}
\source
x=DOT\_PRODUCT(geom(1:nodel,1),fun)\\
y=DOT\_PRODUCT(geom(1:nodel,2),fun)
\end{quote}
This is possible because of the way in which Fortran stores its arrays in memory.
\item[Statement 65 to 67:] The routine {\tt matmul} multiplies the arrays {\tt lder}
and {\tt geom} together. This can be performed using the
intrinsic {\tt MATMUL}. This has been mapped onto the function {\tt MATRIX\_MULTIPLY}
in FELIB90.
\begin{quote}
\source
jac=MATRIX\_MULTIPLY(lder,geom)
\end{quote}
{\tt jac} must be allocated with suitable size and shape.

The routine {\tt matmul} multiplies the arrays {\tt jacin}
and {\tt lder} together. This can be performed using the
intrinsic {\tt MATMUL} which been mapped onto the function {\tt MATRIX\_MULTIPLY}
in FELIB90.
\begin{quote}
\source
gder=MATRIX\_MULTIPLY(jacin,lder)
\end{quote}
{\tt gder} must be allocated with suitable size and shape.
\item[Statements 68 to 75:] The section of code deals with the
construction and assembly the element stiffness matrix {\tt dtpd}. There
two approaches to replacing these statements: firstly by mirroring the FELIB90
routines using intrinsics
\begin{quote}
\source
pd=MATRIX\_MULTIPLY(p,gder)\\
gdert=TRANSPOSE(gder)\\
dtpd=MATRIX\_MULTIPLY(gdert,pd)\\
scvec=fun*src(x,y,strgth)*quot
\end{quote}
remembering that the intrinsic {\tt matnul} is mapped to the FELIB90
routine {\tt MATRIX\_MULTIPLY} or by combining these  three steps into a
single compound statement
\begin{quote}
\source
dtpd=MATRIX\_MULTIPLY(TRANSPOSE(gder),MATRIX\_MULTIPLY(p,gder))
\end{quote}
{\tt dtpd} must be allocated with a suitable size and shape.
The final collections can be performed using
\begin{quote}
\source
elk = elk + dtpd\\
elq = elq + scvec
\end{quote}
instead of using the FELIB90 {\tt matadd} and {\tt vecadd} routines.
\end{description}




\section{The transformation process}
{\tt FELIB} had been developed originally in Fortran~66 - a fixed source form with
all upper case characters - no {\tt IF - THEN - ELSE} constructs and the use of
{\tt GOTO} statements. Moreover {\tt FELIB} was in strict Fortran~66 having been verified
by QA tools such {\tt PFORT} \cite{pfort}.

The basic transformation process had the following steps:
\begin{itemize}
  \item Compile and run on test data. Save results from tests.
	\item Verify Fortran~66 code against standard (using {\tt PFORT} or the {\tt TOOLPACK} tool
	{\tt istpf})
	\item The conforming Fortran~66, which includes many {\tt GOTO} blocks was restructured using
	the {\tt TOOLPACK} tool {\tt istst} and {\tt plusFORT} {\tt SPAG} program. With the correct
	options {\tt SPAG} could transform directly to Fortran~90/95 but it was thought that staging 
	through restructured Fortran~77 would allow some result testing.
	\item Re-compilation and testing of new Fortran~77 using test data.
	\item Transformation of comments etc with {\tt istuc}.
	\item Re-compilation and testing.
	\item Transformation into free format Fortran~90/95 using {\tt SPAG}. Others tools in the 
	{\tt NAGWare} suite could have been used.
	\item Compilation and testing of new source code using test data.
	\end{itemize}
At this point we have transform the Fortran~66 code into free format Fortran~90/95. Along the
route each tool will have detected some problems with the code that required corrections. In
general the corrections were made to the original Fortran~66 and the process repeated.

At this point the structural elements of the re-design were implemented. As the overall
structure and functionality of the example programs and library were not going to changed
the modification at this point were made on a routine by routine basis using a set of
simple edit scripts. For examples: changing the type of variable {\tt SRNAME}, the routine
name, from {\tt DOUBLE PRECISION} to {\tt CHARACTER*6} were simple {\tt awk} scripts.
\begin{quote}
\begin{verbatim}
DOUBLE PRECISION SRNAME
DATA SRNAME /' SRNAME '/
\end{verbatim}
\end{quote}
became
\begin{quote}
\begin{verbatim}
CHARACTER*6 srname = "SRNAME"
\end{verbatim}
\end{quote}
There were many other similar examples.

As with many of the other changes these had to be made on a routine by routine
basis following some basic re-design rules.

\section{Some notes on development tools}
In this section we review some of the software tools used in the re-engineering of {\tt FELIB}. 
These tools were used to process the source code and check the executables.
Regular use of tools such as these will speed development by helping to prevent or to find errors 
in user programs and in making source code easier to read and understand. 
Real benefits can be gained from the use of the tools during maintenance of existing software as the checks help to ensure that modifications are properly applied and that the style of the code remains consistent.

\subsection{Source code transformers}
One of the major stumbling blocks in any re-design or re-engineering process is the
thought that you have thousands of line of code to change. This has been well
recognised in the Fortran community as the use of Fortran~90/95 has developed. For
Fortran~77 many source code analysis and restructuring tools had been
developed: notably through the {\tt TOOLPACK} Project. As Fortran~77 was a fully compliant 
subset of Fortran~90/95 it was generally straightforward to develop source code transformers 
to transform the fixed format Fortran~77 programs to either fixed or free format Fortran~90/95.

Three transformers have been used in this project: the {\tt TOOLPACK} suite,
the {\tt spag} program from the {\tt plusFORT} suite of Polyhedron Software Ltd and 
the {\tt f95} Declaration Standardiser of the {\tt NAGware} 
Tools from the Numerical Algorithms Group Ltd. All these tools will take a fixed format
Fortran~77 program and transform the source. {\tt TOOLPACK} will generate well
structured and formated Fortran~77 from {\it old} Fortran~66 and the other two will transform 
conforming Fortran~77 into reasonable Fortran~90/95 in either a fixed of free format.

However problems do arise if the source Fortran is not conforming. So often elements
of the Unix C pre-processor {\tt cpp} are used as version control constructs in Fortran
programs. {\tt spag} from {\tt plusFORT} was more tolerant of language dialects than {\tt decs95}. 
However both tools were very useful in producing free format Fortran~90/95 code from the
original Fortran~77.

\subsection{The {\tt TOOLPACK} Suite}
For some programs Fortran~66 is the implementation language: the use of {\tt GOTO}s, 
arithmetic {\tt IF}s and computed {\tt GOTO} statements.

{\tt TOOLPACK} is a suite of software tools designed in the 1980s to support the Fortran programmer. 
In this context, a `software tool' is a utility program to assist in the various phases 
of constructing, analysing, testing, adapting, or maintaining a body of Fortran software. 
Typically, the input to such a tool is your Fortran source code. The tool processes this 
and produces output that may have one or both of the following forms:
\begin{itemize}
\item A report that gives an analysis of the input program, e.g. a summary of the types of statements used; this type of tool is called a static analyser.
\item A modified version of the input program; in this case, the tool is called a transformer. An example is a formatter which improves the appearance of the code.
\end{itemize}
In some cases the input may be test data, documentation, or a report generated by a previously applied tool. Tools that assist directly in preparing documents are usually called documentation generation aids. These and other tools serving utility functions all have an important role to play and so, even if they do not process a program directly, they are still regarded as programming aids.

Further examples of the software tools provided include:
\begin{itemize}
\item A text editor with Fortran~77 oriented features.
\item A transforming tool that standardises the declarative part of a Fortran program.
\item An instrumenter that modifies the program by inserting monitoring and other control statements. The instrumented program is then compiled and executed, and data is gathered that is used to generate reports. Execution of an instrumented program is an example of dynamic analysis. 
\end{itemize}
The {\tt TOOLPACK} Suite is public domain and is easily obtained although they are
now of limited use as the community migrates the Fortran 90/95. Some of the tools
contained in {\tt TOOLPACK} have been packaged into the {\tt NAGWare} Fortran~77 Tools.
See 
\begin{quote}
{\tt http://www.nag.co.uk/public/tpack.asp}
\end{quote}
for details.

\subsection{{\tt plusFORT}}

{\tt plusFORT} is a suite of tools for Fortran programmers. The main components are summarised below.

\begin{itemize}
			\item {\tt SPAG}: The primary analysis and restructuring tool.
			\item {\tt GXCHK}:A global static analysis tool.
			\item {\tt CVRANAL}:A coverage analysis reporting tool.
			\item {\tt QMERGE}:A version selection tool.
			\item {\tt QSPLIT}:A small file-splitting utility.
			\item {\tt AUTOMAKE}:A tool for minimal recompilation.
\end{itemize}

{\tt SPAG}, the {\tt plusFORT} restructuring tool, was the one tool that was extensively used. It can unscramble spaghetti Fortran~66 code, and convert it to structured Fortran~77. It also converts back and forth between standard Fortran~77, and code with VAX and Fortran~90/95 extensions such as {\tt DO WHILE}, {\tt ENDDO}, {\tt CYCLE}, {\tt EXIT} and {\tt SELECT CASE}.

{\tt SPAG} does not change the meaning of a program, or even the order in which statements are executed; it does change the way the program logic is written down, making it much easier to understand and maintain. Blocks of code are reordered so that logically related sections are physically close, and jumps in control flow are minimised. {\tt SPAG} may also replicate small code fragments where this improves the re-structured code. {\tt SPAG} computes complexity metrics before and after restructuring. 
{\tt SPAG} contains a powerful code beautifier, with dozens of options controlling spacing, case, labels, indentation, use of {\tt CONTINUE} etc. You can use {\tt SPAG} to switch back and forth between the Fortran~77 and Fortran~90/95 source forms.

There are over 100 configuration options which allow you to customise {\tt SPAG} output to local conventions and requirements. See 
\begin{quote}
{\tt http://www.polyhedron.com}
\end{quote}
 for details.

\subsection{{\tt NAGWare} Tools}

The {\tt NAGWare} Fortran Tools provide users with the ability to analyse and transform Fortran~77 and Fortran~90/95 codes.  These tools can be used in a range of ways: 
\begin{itemize}
	\item Quality assurance standardisation - enforcing coding standards 
	\item Porting to new platforms 
	\item Converting from fixed format Fortran~77 to free format Fortran~95 
	\item Normal day-to-day development 
\end{itemize}
The {\tt NAGWare} Fortran Tools suite consists of the following components: 
\begin{itemize}
	\item {\it {\tt NAGWare} Fortran 95 Tools}:
The {\tt NAGWare} f95 Tools provide analysis and transformational tools that accept as input 
Fortran~77 and fixed or free format Fortran~95. Output from the transformational tools is 
always free format, so these tools are effectively fixed to free format translators. 
This set of tools provides analysis capabilities that include a call graph generator and transformational tools that include a configurable pretty printer, declaration standardiser and precision standardiser. 
  \item{\it{\tt NAGWare} Fortran 77 Tools}:
The {\tt NAGWare} f77 Tools are a collection of tools for processing, analysing and transforming 
Fortran~77 source code. The tools accept as input standard conforming Fortran~77 with some common extensions and output fixed format Fortran~77. So these tools are used where it is not desired to move forward to free format Fortran~95. 
The analysis capabilities which include a portability verifier (standard conformance checker) and 
call graph generator, can be useful as a first step in porting code from Fortran~77 to Fortran~95 
or as an aid to further development work on the Fortran~77 code.
\end{itemize} 
The transformational tools include a configurable pretty printer, declaration standardiser and precision transformer.  See 
\begin{quote}
{\tt http://www.nag.co.uk}
\end{quote}
 for details.

\subsection{Memory checking}
One of the major sources of difficulty in using dynamic arrays in Fortran~90/95
is memory leakage. Without a very careful count of {\tt ALLOCATES} and {\tt DEALLOCATES} 
it is very easy for leaks to arise. This is often true of library software but it is
particularly true of {\tt FELIB90} as it attempts to {\it hide} much of its dynamic memory
management.

During this develop the {\tt memprof} program has been used to help track memory leaks. The
program is freely available over the Internet from 
\begin{quote}
{\tt http://www.gnome.org/projects/memprof/}
\end{quote}
and is easily installed and used. {\tt memprof} is a tool for profiling memory usage and finding 
memory leaks. Its two major features are:
\begin{itemize}
\item It can generate a profile how much memory was allocated by each 
function in your program.
\item It can scan memory and find blocks that you have allocated but are 
no longer referenced anywhere.
\end{itemize}
{\tt memprof} works by pre-loading a library to override the C library's memory allocation 
functions and does not require you to recompile your program.

One advantage {\tt memprof} has over some other similar tools that are available is that it 
has a nice GUI front-end and is relatively easy to use. It appears to work fine on {\tt FELIB90} 
although its diagnostic output, instruction addresses, is not particularly useful. It 
does however give a useful way of indicating the presence of memory leaks.

\subsection{Case transformer: {\tt istuc}}
As noted above {\tt FELIB} was originally in a the fixed, upper case format of Fortran~66. As
a results all the comments in the software were upper case. Given that Fortran~77 and Fortran~90/95
allowed mixed cases it was thought useful to transform the comments into mixed case. The
comments also often referenced variable names. It was thought helpful if these could be left, together
with a few other key words, in upper case.

To make this process as automatic as possible the utilities and libraries of the {\tt TOOLPACK}
suite were used the develop an addition tool to preform this task. {\tt TOOLPACK} provided
all that was need to parse the software, edit and reformed the comment lines and re-construct
the programs source form. {\tt istuc} was used within the {\tt TOOLPACK} command environment 
{\tt istce}. The following {\tt TOOLPACK} script was used to process each file in {\tt FELIB}
\begin{quote}
\begin{verbatim}
lx #&1.f,&1-lx.lst,&1-lx.tkn,&1-lx.cmt
uc &1-lx.tkn,&1-lx.cmt,&1-uc.tkn,&1-uc.cmt,u-words
pl &1-uc.tkn,&1-uc.cmt,#&1.pol,-    
\end{verbatim}
\end{quote}
the script being called thus
\begin{quote}
\begin{verbatim}
ce:com/edit asful
\end{verbatim}
\end{quote}                                                
for the routine {\tt asful.f}. {\tt lx} is the {\tt TOOLPACK} lexical analyser which decomposes
the source code and generates a listing stream ({\tt .lst}), token stream ({\tt .tkn}) and
a comment stream ({\tt .cmt}). {\tt uc} processes the token and comment streams and passes
then on to the {\tt TOOLPACK} polish tool, {\tt pl}, which reconstitutes the source code.
{\tt lx} and {\tt pl} are standard {\tt TOOLPACK} tools.

Although one might wish to preform similar processes in Fortran~90/95 no tool set like that
of {\tt TOOLPACK} is available. Such tasks would then need performing languages such as
{\tt perl} \cite{perl} or {\tt python} \cite{python}.

\section{Conclusions}
In this report we have described the re-design of the Finite Element Library in Fortran~90/95
and explained the major design choices. The overall structure of the library has
been discussed and the use of generic routines and dynamic memory allocation
explored.

We believe that the resulting library will provide a useful addition to the
vast body of Fortran~90/95 computational engineering software available to
the community.

The Fortran~90/95 version of the Finite Element Library will be made available
to the research community through the Group's Web site at:
\begin{quote}
\tt\small
http://www.mathsoft.cse.clrc.ac.uk/felib90
\end{quote}
At present only a small subset of {\tt FELIB} is available in Fortran~90/95 but
this will grow in time. All additional {\tt FELIB90} material such as software and 
documentation, will be made available at this address.

One final comment. Fortran~90/95 has the potential to design and implement
programs in an object oriented approach. The work in this report is a
stepping stone to an object oriented version of the Finite Element Library.

\addcontentsline{toc}{section}{References}
\begin{thebibliography}{99}
\bibitem{felib1} C. Greenough, K. Robinson, {\it The Finite Element Library - Level 1
Documentation - Version 3}, Rutherford Appleton Laboratory, 1990
\bibitem{felib2} C. Greenough, K. Robinson, {\it The Finite Element Library - Level 0 
Documentation - Version 3}, Rutherford Appleton Laboratory, 1990
\bibitem{felib3} C. Greenough, "The Finite Element Library - from design to realisation",
Rutherford Appleton Laboratory, Technical Report, RAL-85-011, 1985
\bibitem{felib4} C. Greenough, C.J. Hunt, PARFEL - An Extension of the NAG/SERC Finite Element
Library for Multi-Processor Message Passing System, Rutherford Appleton Laboratory Report, 
RAL-90-070, 1990
\bibitem{imsmith1} I.M. Smith, {\it Programming the Finite Element Method with Application in
Geo-mechanic}, John Wiley, Chichester, 1982
\bibitem{imsmith2} I.M. Smith, D.V. Griffiths, {\it Programming the Finite Element Method}, 3rd Ed, John Whiley, Chichester, 1998
\bibitem{imsmith3} I.M Smith, A General Purpose System for Finite Element Analysis in Parallel, Engineering Computations, v17, No1, pp75 - 91, 2000
\bibitem{cohen} M. Cohen (ed.), "Information technology - Programming languages - Fortran -
Enhanced data type facilities", ISO/IEC TR 15581(E), ISO, Geneva
\bibitem{toolpack} \textit{TOOLPACK} see {\tt http://www.nag.co.uk/public/tpack.asp}
\bibitem{plusfort} \textit{plusFORT} Reference Manual, Polyhedron Software Ltd\\ (see  {\tt http://www.polyhedron.com/pf/manual/index.html})
\bibitem{nagware}\textit{NAGWare} see {\tt http://www.nag.co.uk/nagware/NQ.asp}
\bibitem{fortran90} M. Metcalf, J. Reid, {\it Fortran 90 Explained}, Oxford University Press
\bibitem{pfort} B.G. Ryder, The PFORT Verifier, Software, Practice and Experience, {\bf 4}, pp359-378, 1974.
\bibitem{perl} R.L. Schwartz, T. Christiansen, {\it Learning Perl}, O'Reilly \& Associates Inc, 1997
\bibitem{python} G. Van Rossum, F.L. Drake Jr (Editor), {\it An Introduction to Python}, Network Theory Ltd., April 2003
\end{thebibliography}
\newpage
\appendix
\addcontentsline{toc}{section}{Appendix A - Reduced Fortran~90/95 version of Seg3p1}
\section*{Appendix A - Reduced Fortran~90/95 version of Seg3p1}
In this section we provide another version of Seg3p1 using more more of the features 
available in Fortran~90/95. We have made some of the substitutions using
the Notes in Section~\ref{notes} and placed multiple statements on
source lines. We have introduced a number of additional {\it intrinsic} functions: 
{\tt matrix\_inverse} and {\tt matrix\_determinant}. This has made the assembly
loop more compact.

It will be noticed that {\tt ALLOCATE} and {\tt DEALLOCATE} statements have been
introduced to manage the dynamic memory arrays. In this program the {\tt ALLOCATE} 
and {\tt DEALLOCATE} statements have been placed near the array's point of use and not
as a vast initialisation block. Also some have been placed within the element loops. This
is not strictly necessary but gives an indication of where they might be needed in a
more complex program using more than one element type.

There are a few other problems given the current operation of some of the routines.
For example {\tt qqua4}: to define storage for {\tt wght} and {\tt abss} the number 
of quadrature  points, {\tt nqp}, must be assumed. This makes returning the value redundant.
Although in this program {\tt qqua4} is placed outside the element loops in a multi-element
type application this would be moved inside the element loop. 

\begin{quote}
\source
\begin{verbatim}     
1    PROGRAM seg3p1
     
     !*****************************************************************
     
     !     Copyright (C) 2003 : CLRC, Rutherford Appleton Laboratory
     !     Chilton, Didcot, Oxfordshire OX11 0QX
     
     ! N.B. The working precision of the current library is held
     !      in the variable wp. This must be used in all REAL
     !      declarations of variables used by FELIB90.
     
     !      The program also uses the standard FELIB90 values for 
     !      nin and nout.
     
     !*****************************************************************
     
2      USE felib90 ! Use FELIB90 all routines
     
3      USE def3p1 ! Use standard SEG3P1 definitions
     
4      IMPLICIT NONE
     
     ! Parameters
     
5      REAL (wp), PARAMETER :: scale = 1.0E+10
     
     ! Local variables
     
6      INTEGER :: bndnod, dif, dimen, dofel, dofnod, elnum, eltyp, hband, i, &
7        iquad, itest, j, nele, node, nodel, nodnum, nqp, totdof, totels, &
8        totnod
9      REAL (wp) :: det, eta, quot, strgth, x, xi, y
     
     ! Allocatable arrays - mesh size dependent
     
10      INTEGER, POINTER :: bnode(:), nf(:,:), eltop(:,:)
11      REAL (wp), POINTER :: bval(:), rhs(:), coord(:,:), sysk(:,:)
     
     ! Intrinsic functions
     
12      INTRINSIC abs
     
     !     Initialisation of POINTERS to main arrays
     
     !     NULLIFY(bnode, nf, eltop, bval, rhs, coord, sysk)
     
13      itest = 0
     
     !     **********************
     !     *                    *
     !     * Input Data Section *
     !     *                    *
     !     **********************
     
     !     Input of nodal geometry - memory for coord automatic
     
14      READ (nin,'(2I5)') totnod, dimen
15      ALLOCATE (coord(totnod,dimen))
16      DO i = 1, totnod
17        READ (nin,'(I5,2F10.0)') node, (coord(node,j),j=1,dimen)
18      END DO
19      CALL prtgeo(coord)
     
     !     Input of element topology - memory for totels automatic
     
20      READ (nin,'(3I5)') eltyp, totels, nodel
21      ALLOCATE (eltop(totels,nodel+2))
22      DO i = 1, totels
23        READ (nin,'(10I5)') elnum, (eltop(elnum,j+2),j=1,nodel)
24        eltop(elnum,1) = eltyp
25        eltop(elnum,2) = nodel
26      END DO
27      CALL prttop(eltop)
     
     !     Input of permeabilities, construction of permeability matrix P
     !     and source strength
     
28      ALLOCATE (p(dimen,dimen))
29      p = 0.0
30      WRITE (nout,'(/A)') 'Permeabilities'
31      READ (nin,'(2F10.0)') (p(i,i),i=1,dimen)
32      WRITE (nout,'(2F10.5)') (p(i,i),i=1,dimen)
     
33      WRITE (nout,'(/A)') 'Source Strength'
34      READ (nin,'(F10.0)') strgth
35      WRITE (nout,'(F10.5)') strgth
     
     !     Input of number of degrees of freedom per node, input of
     !     boundary conditions and construction of nodal freedom array NF
     
36      WRITE (nout,'(/A)') 'Degrees of freedom per node (DOFNOD)'
37      READ (nin,'(I5)') dofnod
38      WRITE (nout,'(I5)') dofnod
     
     !    Input boundary conidtions
     
39      WRITE (nout,'(/A)') 'Boundary Conditions'
40      READ (nin,'(I5)') bndnod
41      WRITE (nout,'(I5)') bndnod
     
42      ALLOCATE (bnode(bndnod),bval(bndnod))
43      bnode = 0.0
44      bval = 0.0
45      DO i = 1, bndnod
46        READ (nin,'(I5,F10.0)') bnode(i), bval(i)
47        WRITE (nout,'(I5,F10.5)') bnode(i), bval(i)
48      END DO
     
     ! Setup nodel freedom array
     
     
49      ALLOCATE (nf(totnod,dofnod))
50      nf = 0
51      totdof = 0
52      DO i = 1, totnod
53        DO j = 1, dofnod
54          totdof = totdof + 1
55          nf(i,j) = totdof
56        END DO
57      END DO
     
     !     Calculation of semi-bandwidth
     
58      CALL bndwth(eltop,nf,hband)
     
     !  ************************************
     !  *                                  *
     !  * System Stiffness Matrix Assembly *
     !  *                                  *
     !  ************************************
     
     
     ! System matrices setup and initalise : rhs, sysk
     
59      ALLOCATE (sysk(totdof,hband),rhs(totdof))
60      sysk = 0.0
61      rhs = 0.0
     
     ! Setup quadrature
     
62      nqp = 4
63      ALLOCATE (wght(nqp),abss(dimen,nqp))
64      CALL qqua4(wght,abss,nqp)
     
     ! Begin main element loop
     
65      DO nele = 1, totels !Loop over all elements
     
66        nodel = eltop(nele,2)
67        dofel = dofnod*nodel
     
     ! Initial memory space for working arrays
     
68        ALLOCATE (jac(dimen,dimen))
69        ALLOCATE (gder(dimen,nodel))
70        ALLOCATE (dtpd(nodel*dofnod,nodel*dofnod))
     
     ! Element matrices setup and initalise: elk, elq, scvec
     
71        ALLOCATE (elk(dofel,dofel),elq(dofel),scvec(dofel))
72        elk = 0.0
73        elq = 0.0
74        scvec = 0.0
     
75        ALLOCATE (geom(dofel,dimen))
76        CALL elgeom(nele,eltop,coord,geom)
     
     !     Integration loop for element stiffness using NQP quadrature
     !     points
     
77        DO iquad = 1, nqp ! Numerical integration
     
     !     Form linear shape function and space derivatives in the local
     !     corrdinates. Transform local derivatives to global coordinate
     !     system
     
78          xi = abss(1,iquad)
79          eta = abss(2,iquad)
     
80          ALLOCATE (fun(nodel),lder(dimen,nodel))
81          CALL quam4(fun,lder,xi,eta)
     
82          x = dot_product(geom(1:nodel,1),fun)
83          y = dot_product(geom(1:nodel,2),fun)
     
84          jac = matrix_multiply(lder,geom)
85          gder = matrix_multiply(matrix_inverse(jac),lder)
86          dtpd = matrix_multiply(transpose(gder),matrix_multiply(p,gder))
     
87          quot = abs(matrix_determinant(jac))*wght(iquad)
88          elk = elk + dtpd*quot
89          elq = elq + fun*src(x,y,strgth)*quot
     
90          DEALLOCATE (fun,lder)
     
91        END DO !Loop over quadrature points - iquad
     
     !     Assembly of system stiffness matrix
     
92        CALL direct(nele,eltop,nf,steer) ! Memory for steer automatic
93        CALL assym(sysk,elk,steer)
94        CALL asrhs(rhs,elq,steer)
     
95        DEALLOCATE (elk,elq,scvec) ! Deallocate element vector & arrays
96        DEALLOCATE (jac,gder,dtpd)
97        DEALLOCATE (geom,geomt)
     
98      END DO !Loop over elements - nele
     
     !     *********************
     !     *                   *
     !     * Equation Solution *
     !     *                   *
     !     *********************
     
     !     Modification of stiffness matrix and right-hand side to
     !     implement boundary conditions
     
99      DO i = 1, bndnod
100        j = bnode(i)
101        sysk(j,hband) = sysk(j,hband)*scale
102        rhs(j) = sysk(j,hband)*bval(i)
103      END DO
     
     !     Solution of system matrix for the nodal values of the
     !     potential
     
104      CALL chosol(sysk,rhs) ! rhs=chosol(sysk,rhs)
     
105      WRITE (nout,'(/A)') 'Nodal Potentials'
106      CALL prtval(rhs,nf)
     
107      STOP
     
108    CONTAINS
     
     
     !*****************************************************************
     ! Source function
     
109      FUNCTION src(x,y,strgth)
     
110        USE felib90
     
111        IMPLICIT NONE
     
     ! Dummy arguments
     
112        REAL (wp) :: src
113        REAL (wp) :: strgth, x, y
     
114        INTENT (IN) strgth, x, y
     
115        src = 0.0D0
116        IF ((x>1.0D0) .AND. (x<2.0D0) .AND. (y>1.0D0) .AND. (y<2.0D0)) &
117          src = strgth
     
118      END FUNCTION src
     
119    END PROGRAM seg3p1
\end{verbatim}

\end{quote}

\end{document}
