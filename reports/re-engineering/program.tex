 \begin{verbatim}
     !*****************************************************************
     
1     PROGRAM seg3p1
     
     !*****************************************************************
     
     !     Copyright (C) 2003 : CLRC, Rutherford Appleton Laboratory
     !     Chilton, Didcot, Oxfordshire OX11 0QX
     
     ! N.B. The working precision of the current library is held
     !      in the variable WP. This must be used in all REAL
     !      declarations of variables used by FELIB90.
     
     !      The program also uses the standard FELIB90 values for 
     !      nin and nout.
     
     !*****************************************************************
     
2       USE felib90 ! Use FELIB90 all routines
3       USE def3p1 ! Use standard SEG3P1 definitions
4       IMPLICIT NONE
     
     ! Parameters
     
5       REAL (WP), PARAMETER :: scale = 1.0E+10
     
     ! Local variables
     
6       INTEGER :: bndnod, dimen, dofel, dofnod, hband, i, iquad, itest, j, &
7         nele, nodel, nqp, totdof, totels, totnod
8       REAL (WP) :: det, eta, quot, strgth, x, xi, y
     
     ! Allocatable arrays - mesh size dependent - user defined in the data
     
9       INTEGER, POINTER :: bnode(:), nf(:,:), eltop(:,:)
10      REAL (WP), POINTER :: bval(:), rhs(:), coord(:,:), sysk(:,:)
     
     ! Intrinsic functions
     
11      INTRINSIC abs
     
     !     Initialisation of POINTERS to main arrays
     
12      NULLIFY (bnode,nf,eltop)
13      NULLIFY (bval,rhs,coord,sysk)
     
     !     Set error checking flag
     
14      itest = 0
     
     !     **********************
     !     *                    *
     !     * Input Data Section *
     !     *                    *
     !     **********************
     
     !     Input of nodal geometry
     
15      CALL getgeo(coord,totnod,dimen)
16      CALL prtgeo(coord)
     
     !     Input of element topology
     
17      CALL gettop(eltop,totels)
18      CALL prttop(eltop)
     
     !     Input of permeabilities, construction of permeability matrix P
     !     and source strength
     
19      CALL matnul(p,dimen,dimen)
20      WRITE (nout,'(/A)') 'Permeabilities'
21      READ (nin,'(2F10.0)') (p(i,i),i=1,dimen)
22      WRITE (nout,'(2F10.5)') (p(i,i),i=1,dimen)
     
23      WRITE (nout,'(/A)') 'Source Strength'
24      READ (nin,'(F10.0)') strgth
25      WRITE (nout,'(F10.5)') strgth
     
     !     Input of number of degrees of freedom per node, input of
     !     boundary conditions and construction of nodal freedom array NF
     
26      WRITE (nout,'(/A)') 'Degrees of freedom per node (DOFNOD)'
27      READ (nin,'(I5)') dofnod
28      WRITE (nout,'(I5)') dofnod
     
     !    Input boundary conidtions
     
29      WRITE (nout,'(/A)') 'Boundary Conditions'
30      READ (nin,'(I5)') bndnod
31      WRITE (nout,'(I5)') bndnod
     
32      CALL vecnul(bnode,bndnod)
33      CALL vecnul(bval,bndnod)
34      DO i = 1, bndnod
35        READ (nin,'(I5,F10.0)') bnode(i), bval(i)
36        WRITE (nout,'(I5,F10.5)') bnode(i), bval(i)
37      END DO
     
     ! Setup nodel freedom array
     
38      CALL matnul(nf,totnod,dofnod)
39      totdof = 0
40      DO i = 1, totnod
41        DO j = 1, dofnod
42          totdof = totdof + 1
43          nf(i,j) = totdof
44        END DO
45      END DO
     
     !     Calculation of semi-bandwidth
     
46      CALL bndwth(eltop,nf,hband)
     
     !  ************************************
     !  *                                  *
     !  * System Stiffness Matrix Assembly *
     !  *                                  *
     !  ************************************
     
     
     ! System matrices setup and initalise : rhs, sysk
     
47      CALL matnul(sysk,totdof,hband)
48      CALL vecnul(rhs,totdof) 
     
     ! Setup quadrature
     
49      CALL qqua4(wght,abss,nqp)
     
     ! Begin main element loop
     
50      DO nele = 1, totels !Loop over all elements
     
51        nodel = eltop(nele,2)
52        dofel = dofnod*nodel
53        CALL elgeom(nele,eltop,coord,geom)
     
     !     Integration loop for element stiffness using NQP quadrature
     !     points
     
54        CALL matnul(elk,dofel,dofel) 
55        CALL vecnul(elq,dofel)
56        CALL vecnul(scvec,dofel)
     
57        DO iquad = 1, nqp !Numerical integration
     
     !     Form linear shape function and space derivatives in the local
     !     corrdinates. Transform local derivatives to global coordinate
     !     system
     
58          xi = abss(1,iquad)
59          eta = abss(2,iquad)
60          CALL quam4(fun,lder,xi,eta)
     
61          CALL matran(geom,geomt)
62          CALL matvec(geomt,fun,xy) 
63          x = xy(1)
64          y = xy(2)
     
65          CALL matmul(lder,geom,jac) 
66          CALL matinv(jac,jacin,det)
67          CALL matmul(jacin,lder,gder)
     
     !     Formation of element stiffness ELK
     
68          CALL matmul(p,gder,pd)
69          CALL matran(gder,gdert)
70          CALL matmul(gdert,pd,dtpd)
     
71          quot = abs(det)*wght(iquad)
72          dtpd = dtpd*quot
73          scvec = fun*src(x,y,strgth)*quot
     
74          CALL matadd(elk,dtpd)
75          CALL vecadd(elq,scvec)
     
76        END DO !Loop over quadrature points - iquad
     
     !     Assembly of system stiffness matrix
     
77        CALL direct(nele,eltop,nf,steer)
78        CALL assym(sysk,elk,steer)
79        CALL asrhs(rhs,elq,steer)
     
80      END DO !Loop over elements - nele
     
     !     *********************
     !     *                   *
     !     * Equation Solution *
     !     *                   *
     !     *********************
     
     !     Modification of stiffness matrix and right-hand side to
     !     implement boundary conditions
     
81      DO i = 1, bndnod
82        j = bnode(i)
83        sysk(j,hband) = sysk(j,hband)*scale
84        rhs(j) = sysk(j,hband)*bval(i)
85      END DO
     
     !     Solution of system matrix for the nodal values of the
     !     potential
     
86      CALL chosol(sysk,rhs)
     
87      WRITE (nout,'(/A)') 'Nodal Potentials'
88      CALL prtval(rhs,nf)
     
89      STOP
     
90    CONTAINS
     
     !*****************************************************************
     ! Source function
     
91      FUNCTION src(x,y,strgth)
     
92        USE felib90
     
93        IMPLICIT NONE
     
     ! Dummy arguments
     
94        REAL (wp) :: src
95        REAL (wp) :: strgth, x, y
96        INTENT (IN) strgth, x, y
     
97        src = 0.0D0
98        IF ((x>1.0D0) .AND. (x<2.0D0) .AND. (y>1.0D0) .AND. (y<2.0D0)) &
99          src = strgth
     
100      END FUNCTION src
     
101    END PROGRAM seg3p1
\end{verbatim}