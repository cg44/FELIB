\begin{verbatim}     
1    PROGRAM seg3p1
     
     !*****************************************************************
     
     !     Copyright (C) 2003 : CLRC, Rutherford Appleton Laboratory
     !     Chilton, Didcot, Oxfordshire OX11 0QX
     
     ! N.B. The working precision of the current library is held
     !      in the variable wp. This must be used in all REAL
     !      declarations of variables used by FELIB90.
     
     !      The program also uses the standard FELIB90 values for 
     !      nin and nout.
     
     !*****************************************************************
     
2      USE felib90 ! Use FELIB90 all routines
     
3      USE def3p1 ! Use standard SEG3P1 definitions
     
4      IMPLICIT NONE
     
     ! Parameters
     
5      REAL (wp), PARAMETER :: scale = 1.0E+10
     
     ! Local variables
     
6      INTEGER :: bndnod, dif, dimen, dofel, dofnod, elnum, eltyp, hband, i, &
7        iquad, itest, j, nele, node, nodel, nodnum, nqp, totdof, totels, &
8        totnod
9      REAL (wp) :: det, eta, quot, strgth, x, xi, y
     
     ! Allocatable arrays - mesh size dependent
     
10      INTEGER, POINTER :: bnode(:), nf(:,:), eltop(:,:)
11      REAL (wp), POINTER :: bval(:), rhs(:), coord(:,:), sysk(:,:)
     
     ! Intrinsic functions
     
12      INTRINSIC abs
     
     !     Initialisation of POINTERS to main arrays
     
     !     NULLIFY(bnode, nf, eltop, bval, rhs, coord, sysk)
     
13      itest = 0
     
     !     **********************
     !     *                    *
     !     * Input Data Section *
     !     *                    *
     !     **********************
     
     !     Input of nodal geometry - memory for coord automatic
     
14      READ (nin,'(2I5)') totnod, dimen
15      ALLOCATE (coord(totnod,dimen))
16      DO i = 1, totnod
17        READ (nin,'(I5,2F10.0)') node, (coord(node,j),j=1,dimen)
18      END DO
19      CALL prtgeo(coord)
     
     !     Input of element topology - memory for totels automatic
     
20      READ (nin,'(3I5)') eltyp, totels, nodel
21      ALLOCATE (eltop(totels,nodel+2))
22      DO i = 1, totels
23        READ (nin,'(10I5)') elnum, (eltop(elnum,j+2),j=1,nodel)
24        eltop(elnum,1) = eltyp
25        eltop(elnum,2) = nodel
26      END DO
27      CALL prttop(eltop)
     
     !     Input of permeabilities, construction of permeability matrix P
     !     and source strength
     
28      ALLOCATE (p(dimen,dimen))
29      p = 0.0
30      WRITE (nout,'(/A)') 'Permeabilities'
31      READ (nin,'(2F10.0)') (p(i,i),i=1,dimen)
32      WRITE (nout,'(2F10.5)') (p(i,i),i=1,dimen)
     
33      WRITE (nout,'(/A)') 'Source Strength'
34      READ (nin,'(F10.0)') strgth
35      WRITE (nout,'(F10.5)') strgth
     
     !     Input of number of degrees of freedom per node, input of
     !     boundary conditions and construction of nodal freedom array NF
     
36      WRITE (nout,'(/A)') 'Degrees of freedom per node (DOFNOD)'
37      READ (nin,'(I5)') dofnod
38      WRITE (nout,'(I5)') dofnod
     
     !    Input boundary conidtions
     
39      WRITE (nout,'(/A)') 'Boundary Conditions'
40      READ (nin,'(I5)') bndnod
41      WRITE (nout,'(I5)') bndnod
     
42      ALLOCATE (bnode(bndnod),bval(bndnod))
43      bnode = 0.0
44      bval = 0.0
45      DO i = 1, bndnod
46        READ (nin,'(I5,F10.0)') bnode(i), bval(i)
47        WRITE (nout,'(I5,F10.5)') bnode(i), bval(i)
48      END DO
     
     ! Setup nodel freedom array
     
     
49      ALLOCATE (nf(totnod,dofnod))
50      nf = 0
51      totdof = 0
52      DO i = 1, totnod
53        DO j = 1, dofnod
54          totdof = totdof + 1
55          nf(i,j) = totdof
56        END DO
57      END DO
     
     !     Calculation of semi-bandwidth
     
58      CALL bndwth(eltop,nf,hband)
     
     !  ************************************
     !  *                                  *
     !  * System Stiffness Matrix Assembly *
     !  *                                  *
     !  ************************************
     
     
     ! System matrices setup and initalise : rhs, sysk
     
59      ALLOCATE (sysk(totdof,hband),rhs(totdof))
60      sysk = 0.0
61      rhs = 0.0
     
     ! Setup quadrature
     
62      nqp = 4
63      ALLOCATE (wght(nqp),abss(dimen,nqp))
64      CALL qqua4(wght,abss,nqp)
     
     ! Begin main element loop
     
65      DO nele = 1, totels !Loop over all elements
     
66        nodel = eltop(nele,2)
67        dofel = dofnod*nodel
     
     ! Initial memory space for working arrays
     
68        ALLOCATE (jac(dimen,dimen))
69        ALLOCATE (gder(dimen,nodel))
70        ALLOCATE (dtpd(nodel*dofnod,nodel*dofnod))
     
     ! Element matrices setup and initalise: elk, elq, scvec
     
71        ALLOCATE (elk(dofel,dofel),elq(dofel),scvec(dofel))
72        elk = 0.0
73        elq = 0.0
74        scvec = 0.0
     
75        ALLOCATE (geom(dofel,dimen))
76        CALL elgeom(nele,eltop,coord,geom)
     
     !     Integration loop for element stiffness using NQP quadrature
     !     points
     
77        DO iquad = 1, nqp ! Numerical integration
     
     !     Form linear shape function and space derivatives in the local
     !     corrdinates. Transform local derivatives to global coordinate
     !     system
     
78          xi = abss(1,iquad)
79          eta = abss(2,iquad)
     
80          ALLOCATE (fun(nodel),lder(dimen,nodel))
81          CALL quam4(fun,lder,xi,eta)
     
82          x = dot_product(geom(1:nodel,1),fun)
83          y = dot_product(geom(1:nodel,2),fun)
     
84          jac = matrix_multiply(lder,geom)
85          gder = matrix_multiply(matrix_inverse(jac),lder)
86          dtpd = matrix_multiply(transpose(gder),matrix_multiply(p,gder))
     
87          quot = abs(matrix_determinant(jac))*wght(iquad)
88          elk = elk + dtpd*quot
89          elq = elq + fun*src(x,y,strgth)*quot
     
90          DEALLOCATE (fun,lder)
     
91        END DO !Loop over quadrature points - iquad
     
     !     Assembly of system stiffness matrix
     
92        CALL direct(nele,eltop,nf,steer) ! Memory for steer automatic
93        CALL assym(sysk,elk,steer)
94        CALL asrhs(rhs,elq,steer)
     
95        DEALLOCATE (elk,elq,scvec) ! Deallocate element vector & arrays
96        DEALLOCATE (jac,gder,dtpd)
97        DEALLOCATE (geom,geomt)
     
98      END DO !Loop over elements - nele
     
     !     *********************
     !     *                   *
     !     * Equation Solution *
     !     *                   *
     !     *********************
     
     !     Modification of stiffness matrix and right-hand side to
     !     implement boundary conditions
     
99      DO i = 1, bndnod
100        j = bnode(i)
101        sysk(j,hband) = sysk(j,hband)*scale
102        rhs(j) = sysk(j,hband)*bval(i)
103      END DO
     
     !     Solution of system matrix for the nodal values of the
     !     potential
     
104      CALL chosol(sysk,rhs) ! rhs=chosol(sysk,rhs)
     
105      WRITE (nout,'(/A)') 'Nodal Potentials'
106      CALL prtval(rhs,nf)
     
107      STOP
     
108    CONTAINS
     
     
     !*****************************************************************
     ! Source function
     
109      FUNCTION src(x,y,strgth)
     
110        USE felib90
     
111        IMPLICIT NONE
     
     ! Dummy arguments
     
112        REAL (wp) :: src
113        REAL (wp) :: strgth, x, y
     
114        INTENT (IN) strgth, x, y
     
115        src = 0.0D0
116        IF ((x>1.0D0) .AND. (x<2.0D0) .AND. (y>1.0D0) .AND. (y<2.0D0)) &
117          src = strgth
     
118      END FUNCTION src
     
119    END PROGRAM seg3p1
\end{verbatim}
