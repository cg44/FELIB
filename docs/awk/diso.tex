\documentstyle[twoside,11pt]{felib0}
\release{4}
\date{September 90}
\newcommand{\gt}{>}
\newcommand{\lt}{<}
\routine{DISO}
\title{Stress-Strain Matrix}
\begin{document}
\begin{purpose}
DISO forms the 6x6 stress-strain matrix for use in three
dimensional isotropic problems.
\end{purpose}
\begin{specification}
      SUBROUTINE DISO(D,ID,JD,E,NU,NUMSS,ITEST)\\
C     INTEGER ID,JD,NUMSS,ITEST\\
C     \real D,E,NU\\
C     DIMENSION D(ID,JD)\\
\end{specification}
\begin{description}
The routine DISO forms the stress-strain matrix D for use
in three dimensional isotropic problems. The matrix is 6x6
and the elements are given below.
where E is the Young's Modulus and   the Poisson's ratio for
the problem.
\end{description}
\begin{references}
\ref{1} BATHE, K-J., WILSON, E.L.\\
      Numerical Methods in Finite Element Analysis, p 110
      Prentice-Hall, 1976.
\end{references}
\begin{parameters}
\param{D} {{real}} array of DIMENSION (ID,JD) where ID $\leq$ 6 and JD $\leq$ 6.\\
      On successful exit, D contains the stress-strain matrix.
\param{ID} INTEGER.\\
      On entry, ID specifies the first dimension of array D as
      declared in the calling (sub)program.
      Unchanged on exit.
\param{JD} INTEGER.\\
      On entry, JD specifies the second dimension of array D as
      declared in the calling (sub)program.
      Unchanged on exit.
\param{E} {{real}}.\\
      On entry, E specifies the value of Young's Modulus to be
      used in calculating the stress-strain matrix.
      Unchanged on exit.
\param{NU} {{real}}.\\
      On entry, NU specifies the value of Poisson's ratio to be
      used in calculating the stress-strain matrix.
      Unchanged on exit.
\param{NUMSS} INTEGER.\\
      On successful exit, NUMSS specifies the order of the
      stress-strain matrix. NUMSS = 6 in this case.
\input itest.tex
\end{parameters}
\begin{indicators}
\error{1} On entry, ID or JD $\lt$ 6.
\error{2} On entry, 0 $\geq$ NU $\geq$ 0.5.
\end{indicators}
\begin{routines}
This routine uses the Level 0 Library routines MATNUL and ERRMES.
\end{routines}
\begin{timing}
Not available.
\end{timing}
\begin{storage}
There are no internally declared arrays.
\end{storage}
\begin{accuracy}
{{Basic precision}} arithmetic is used.
\end{accuracy}
\begin{comments}
None.
\end{comments}
\end{document}
