Library routines can be conveniently divided
up into a number of sections. Although the routines are
presented in alphabetical order by routine name,
below the routines are divided into sections
depending upon their operation.

%\medskip\noindent     %PAPER
%The sections are:     %PAPER
%\begin{enumerate}     %PAPER
%\item Standard Vector and Matrix Operations     %PAPER
%\item Equation Solvers     %PAPER
%\item Eigenvalue and Eigenvector Extraction     %PAPER
%\item Data Handling     %PAPER
%\item Numerical Integration     %PAPER
%\item Shape Functions     %PAPER
%\item Strain-Displacement Matrices     %PAPER
%\item Stress-Strain Matrices     %PAPER
%\item Assembly Routines     %PAPER
%\item Utility Routines     %PAPER
%\end{enumerate}     %PAPER
%\newpage     %PAPER
%\section*{Standard Vector and Matrix Handling}   %PAPER
\subsection*{Standard Vector and Matrix Handling} %HTML
\begin{list}{}{\leftmargin=57pt \labelwidth=55pt}
\item[CMTNUL \hfill] -- Initialises a complex matrix to zero
\item[CVCNUL \hfill] -- Initialises a complex vector to zero
\item[DYAD \hfill]   -- Forms the dyad of two vectors
\item[MATADD \hfill]  -- Adds two full matrices together
\item[MATCOP \hfill] -- Copies one full matrix into another
\item[MATIDN \hfill] -- Initialises a matrix to the identity matrix
\item[MATINV \hfill] -- Inverts a full matrix (for limited sizes)
\item[MATMUL \hfill] -- Multiplies two matrices together
\item[MATNUL \hfill] -- Initialises a full matrix to zeros
\item[MATRAN \hfill] -- Forms the transpose of a full matrix
\item[MATSUB \hfill] -- Subtracts one full matrix from another
\item[MATVEC \hfill] -- Multiplies a full matrix by a vector
\item[MVSYB \hfill]  -- Multiplies a symmetric band matrix by a vector
\item[MVUSB \hfill]  -- Multiplies a unsymmetric band matrix by a vector
\item[SCAPRD \hfill] -- Forms the scalar product of two vectors
\item[VECADD \hfill] -- Adds two vectors together
\item[VECCOP \hfill] -- Copies one vector into another
\item[VECMAT \hfill] -- Pre-multiplies a full matrix by a vector
\item[VECNUL \hfill] -- Initialises a vector to zeros
\item[VECSUB \hfill] -- Subtracts one vector from another
\item[VMSYB \hfill]  -- Pre-multiplies a symmetric band matrix by a vector
\item[VMUSB \hfill]  -- Pre-multiplies an unsymmetric band matrix by a vector
\end{list}
%\section*{Equation Solvers} %PAPER
\subsection*{Equation Solvers} %HTML
\begin{list}{}{\leftmargin=57pt \labelwidth=55pt}
\item[CSYRDN \hfill] -- Complex symmetric decomposition into triangular matrices
\item[CSYSOL \hfill] -- Solves a complex symmetric linear system
\item[CSYSUB \hfill] -- Complex forward and backward substitution
\item[CHOBAK \hfill] -- Choleski backward substitution
\item[CHOFWD \hfill] -- Choleski forward substitution
\item[CHORDN \hfill] -- Choleski decomposition into triangular matrices
\item[CHOSOL \hfill] -- Solves a system of linear equations using Choleski reduction
\item[GAURDN \hfill] -- Gaussian decomposition into upper and lower triangles
\item[GAUSOL \hfill] -- Solves a system of linear equations using Gaussian reduction
\item[GAUSUB \hfill] -- Gaussian forward and backward substitution
\end{list}
%\section*{Eigenvalue and Eigenvector Extraction} %PAPER
\subsection*{Eigenvalue and Eigenvector Extraction} %HTML
\begin{list}{}{\leftmargin=57pt \labelwidth=55pt}
\item[HOUSE \hfill]  -- Householder reduction to tridiagonal form
\item[JACO \hfill]   -- Reduction to tridiagonal form by Jacobi rotations
\item[QLVAL \hfill]  -- Eigenvalue extraction using QL transformations
\item[QLVEC \hfill]  -- Eigenvalue and eigenvector extraction using QL transformations
\end{list}
%\section*{Data Handling} %PAPER
\subsection*{Data Handling} %HTML
\begin{list}{}{\leftmargin=57pt \labelwidth=55pt}
\item[CPRTMT \hfill] -- Prints a complex matrix in a standard form
\item[CPRTVC \hfill] -- Prints a complex vector in a standard form
\item[CPRTVL \hfill] -- Prints a complex nodal values in a standard form
\item[PRTGEO \hfill] -- Prints the mesh geometry in a standard format
\item[PRTMAT \hfill] -- Prints a matrix in standard format
\item[PRTTOP \hfill] -- Prints the element topology in a standard format
\item[PRTVAL \hfill] -- Prints the nodal values in a standard format
\item[PRTVEC \hfill] -- Prints a vector in standard format
\end{list}
%\section*{Numerical Integration}   %PAPER
\subsection*{Numerical Integration}    %HTML
\begin{list}{}{\leftmargin=57pt \labelwidth=55pt}
\item[BQBRK \hfill]  -- Forms quadrature rule for line integration on hexahedrals
\item[BQTRI \hfill]  -- Forms quadrature rule for line integration on triangles
\item[BQQUA \hfill]  -- Forms quadrature rule for line integration on quadrilaterals
\item[LINTRI \hfill] -- Calculates a unit length on the boundaries of triangles
\item[LINQUA \hfill] -- Calculates a unit length on the boundaries of quadrilaterals
\item[QBRK6 \hfill]  -- Six-point quadrature rule for cube
\item[QBRK8 \hfill]  -- Eight-point quadrature rule for cube
\item[QLIN2 \hfill]  -- Two-point, one-dimensional quadrature
\item[QLIN3 \hfill]  -- Three-point, one-dimensional quadrature
\item[QQUA4 \hfill]  -- Four-point quadrature rule for rectangular region
\item[QQUA9 \hfill]  -- Nine-point quadrature rule for rectangular region
\item[QTET4 \hfill]  -- Four-point quadrature rule for tetrahedral region
\item[QTRI4 \hfill]  -- Four-point quadrature rule for triangular region
\item[QTRI7 \hfill]  -- Seven-point quadrature rule for triangular region
\item[QWDG6 \hfill]  -- Six-point quadrature rule for wedge-shaped region
\item[QWDG8 \hfill]  -- Eight-point quadrature rule for wedge-shaped region
\item[SURBRK \hfill] -- Calculates a unit of area on boundary faces
\end{list}
%\section*{Shape Functions} %PAPER
\subsection*{Shape Functions} %HTML
\begin{list}{}{\leftmargin=57pt \labelwidth=55pt}
\item[BRK8 \hfill]   -- Shape functions for eight-noded brick element
\item[BRK20 \hfill]  -- Shape functions for twenty-noded brick element
\item[BRK32 \hfill]  -- Shape functions for thirty-two-noded brick element
\item[QUAM4 \hfill]  -- Shape functions for four-noded quadrilateral membrane element
\item[QUAM8 \hfill]  -- Shape functions for eight-noded quadrilateral membrane element
\item[QUAM12 \hfill] -- Shape functions for twelve-noded quadrilateral membrane element
\item[ROD2 \hfill]   -- Shape functions for two-noded line element
\item[ROD3 \hfill]   -- Shape functions for three-noded line element
\item[ROD4 \hfill]   -- Shape functions for four-noded line element
\item[TET4 \hfill]   -- Shape functions for four-noded tetrahedral element
\item[TET10 \hfill]  -- Shape functions for ten-noded tetrahedral element
\item[TET20 \hfill]  -- Shape functions for twenty-noded tetrahedral element
\item[TRIM3 \hfill]  -- Shape functions for three-noded triangular membrane element
\item[TRIM6 \hfill]  -- Shape functions for six-noded triangular membrane element
\item[TRIM10 \hfill] -- Shape functions for ten-noded triangular membrane element
\item[WDG6 \hfill]   -- Shape functions for six-noded wedge element
\item[WDG15 \hfill]  -- Shape functions for fifteen-noded wedge element
\end{list}
%\section*{Strain-Displacement Matrices} %PAPER
\subsection*{Strain-Displacement Matrices}  %HTML
\begin{list}{}{\leftmargin=57pt \labelwidth=55pt}
\item[B2C2 \hfill]   -- Forms strain-displacement matrix for 2-D plane elasticity
\item[B2P2 \hfill]   -- Forms strain-displacement matrix for axisymmetric elasticity
\item[B3C3 \hfill]   -- Forms strain-displacement matrix for 3-D elasticity
\end{list}
%\section*{Stress-Strain Matrices}   %PAPER
\subsection*{Stress-Strain Matrices}   %HTML
\begin{list}{}{\leftmargin=57pt \labelwidth=55pt}
\item[DAXI \hfill]   -- Forms stress-strain matrix for axisymmetric elasticity
\item[DISO \hfill]   -- Forms stress-strain matrix for 3-D elasticity
\item[DPLT \hfill]   -- Forms stress-strain matrix for 2-D plate bending
\item[DPSN \hfill]   -- Forms stress-strain matrix for plane strain
\item[DPSS \hfill]   -- Forms stress-strain matrix for plane stress
\end{list}
%\section*{Assembly Routines}    %PAPER
\subsection*{Assembly Routines}   HTML%
\begin{list}{}{\leftmargin=57pt \labelwidth=55pt}
\item[IASRHS \hfill] -- Assembles the imaginary part of a complex right-hand side
\item[IASSYM \hfill] -- Assembles the imaginary part of a symmetric complex system matrix
\item[IASUSM \hfill] -- Assembles the imaginary part of a unsymmetric complex system matrix
\item[ASFUL \hfill]  -- Assembles a full system matrix
\item[ASFULG \hfill] -- General assembly of a full system matrix
\item[ASLMS \hfill]  -- Assembles a system lumped mass matrix
\item[ASRHS \hfill]  -- Assembles the right-hand side of a system
\item[ASSYM \hfill]  -- Assembles an symmetric system matrix
\item[ASSYMG \hfill] -- General assembly of a symmetric system matrix
\item[ASUSM \hfill]  -- Assembles a unsymmetric system matrix
\item[ASUSMG \hfill] -- General assembly of an unsymmetric system matrix
\item[RASRHS \hfill] -- Assembles the real part of a complex right-hand side
\item[RASSYM \hfill] -- Assembles the real part of a symmetric complex system matrix
\item[RASUSM \hfill] -- Assembles the real part of a unsymmetric complex system matrix
\end{list}
%\section*{Utility Routines}   %PAPER
\subsection*{Utility Routines}      %HTML
\begin{list}{}{\leftmargin=57pt \labelwidth=55pt}
\item[ADUNIT \hfill] -- Control the advisory message unit number
\item[BNDWTH \hfill] -- Calculates semi-bandwidth of system matrix
\item[DCSTRI \hfill] -- Calculates the direction cosines for triangular elements
\item[DCSQUA \hfill] -- Calculates the direction cosines for quadrilateral elements
\item[DCSBRK \hfill] -- Calculates the direction cosines for hexahedral elements
\item[DIRECT \hfill] -- Construction of steering vector for system matrix assembly
\item[ELGEOM \hfill] -- Construction of element geometry array for an element
\item[ERRMES \hfill] -- Controls processing of error conditions
\item[ERUNIT \hfill] -- Control the error message unit number
\item[FORMNF \hfill] -- Forms nodal freedom array
\item[FREDIF \hfill] -- Calculates of maximum freedom number difference for an element
\item[NORM \hfill]   -- Computes the $L^2$ norm of a vector
\item[SHAPFN \hfill] -- Forms shape function matrix {\bf~N}
\item[SELECT \hfill] -- Constructs an element solution vector
\item[SIDENO \hfill] -- Forms list of element and side numbers from boundary node list
\item[UPDATE \hfill] -- Updates solution vector with system solution
\end{list}
